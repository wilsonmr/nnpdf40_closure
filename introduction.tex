\section{Introduction}

In an NNPDF fit to experimental data, a set of replicas are fitted to pseudodata which
is generated according the experimental central values and uncertainties. A
successful fit is realised if the resulting replicas agree with the underlying
law within uncertainties. Since the underlying law is usually unknown we cannot
directly measure if this is the case.

We can, however, test the methodology in
a fit to artificial data, which is generated from theory predictions from an input PDF
and check that we agree with the input PDF within uncertainties. This is the basis
of a closure test, which has already been used to test to validity of previous
iterations of the NNPDF methodology. Here we aim to refine some of the pre-existing
closure test estimators and with the help of fast fitting methodology perform
a more extensive study of how faithful our uncertainties are.

In a fit to experimental data, for a given replica we minimise the $\chi^2$
between the predictions obtained from that replica and the pseudodata which is
obtained by adding some noise to the experimental central values. After fitting
many of these replicas we can construct an ensemble which represents a sample
from the probability distribution of the PDF given the data. In this
way we propagate the various sources of uncertainty involved with fitting PDFs
into the functional PDF space. If we consider experimental data which we assume
to be multigaussian then the experimental central values, $\levone$, are given by
\begin{equation}
    \levone_{i} = \law_i + \shift_i
\end{equation}
in other words, the experimental values have been shifted away from the true
values given by nature, $\law$, by some shift, $\shift$. The shift is drawn from
the multigaussian $\mathcal{N}(0, \cov)$ where $\cov$ is the experimental covariance
matrix. The fitted pseudodata is obtained by adding Monte Carlo noise, $\noise^{\repind}$,
on top of the experimental central values
\begin{equation}
    \levtwo^{\repind}_{i} = \law_i + \shift_i + \noise^{\repind}_{i},
\end{equation}
where the replica index $k$ refers to each replica having different noise drawn
independently from $\mathcal{N}(0, \cov)$. The $\chi^2$ which is minimised for
replica $k$ is then given by
\begin{equation}
    {\chi^2}^{\repind} = \frac{1}{\ndata} \sum_{ij} \diffreptwo_i \invcov{ij} \diffreptwo_j.
\end{equation}
After fitting many replicas, the quality of a fit is often determined by
considering the $\chi^2$ between the experimental central values and the
central value of the replica theory predictions
\begin{equation}
    \chi^2 = \frac{1}{\ndata} \sum_{ij} \diffcentone_i \invcov{ij} \diffcentone_j,
\end{equation}
where $\erep{\cdot}$ denotes the mean value across replicas. This $\chi^2$ is a
measure of the difference between the central prediction and the experimental
central values in units of the covariance.

In a fit to
experimental data we are limited to this quanitity because we don't have knowledge
of the underlying theory predictions of nature $\law$. In a closure test we use
a pre-existing PDF as an input to obtain $\law$ and then generate both
$\shift$ and $\noise$ from $\mathcal{N}(0, \cov)$, emulating the different levels
of data. The underlying theory predictions are referred to as level zero data.
We refer to the shifted central values as level one data, the underlying
law plus a level one shift. The pseudodata which a given replica fits is then
referred to as level two data.

\section{Datasets}

Want to tabulate which datasets were fitted and which were not. If we keep new
and old process types in out of sample then we should also define this here.