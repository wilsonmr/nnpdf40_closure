\section{Introduction}
\label{sec:Intro}

Inverse problems are the typical example of inference where a model is sought
starting from a finite-dimensional set of experimental observations. These
problems are notoriously difficult, and often require trying to guess a
continuous function, \ie\ an element of an infinite dimensional space, from a
finite amount of data. As emphasised by Hadamard a long time ago, it is easy to
end up in a situation where we deal with ill-posed problems, in the sense that
the solution may not exist, may not be unique, or may be unstable under small
variations of the input data. The determination of parton distributions from
experimental data, or the reconstruction of spectral densities from lattice QCD
simulations, are just two examples where these problems arise in particle
physics. In all these cases, finding a robust solution becomes very challenging,
if not impossible, making these questions all the more urgent, especially at a
time when precision studies are the ultimate challenge in order to highlight
divergences from the Standard Model. 

A Bayesian approach provides an apter tool for addressing inverse problems.
Starting from a prior distribution for the model, which encodes our theoretical
knowledge or prejudice, basic statistical tools allow us to determine the
posterior distribution of the solution, {\it after taking into account a set of
experimental observations.} The prior and posterior probabilities encode our
knowledge about the model before and after incorporating the experimental
results. There are multiple advantages to this formulation: the inverse problem
is well defined, the prior distribution ensures that all the assumptions are
explicit, and it allows to regulate the likelihood distribution. 

Note also that probability measures can be defined in infinite-dimensional
spaces, In cases where we are looking to determine a continuous function, the
Bayesian approach allows, at least in principle, to address the problem directly
in terms of the posterior probability measure of the model function. It is often
convenient for practical reasons to define a parametrization of the model
function in terms of a finite, albeit large, number of parameters and reduce the
problem to a finite dimensional one. The price to pay for this simplification is
the introduction of some bias, which needs to be understood and possibly
quantified. An expressive parametrization clearly helps in this case. The
Bayesian approach is also well suited to include potentially inconsistent data
in a systematic way. We will briefly touch upon this aspect below. 

A Bayesian approach to inverse problems has been actively developed by
mathematicians for a long time, and this development has accelerated quickly in
the last decade. In this paper we aim at using the existing framework and adapt
it to analyse the fits of parton distribution functions obtained by the NNPDF
collaboration. We will review the formalism in Sect. 2, trying to define the
notation that we will be used in the rest of the paper. We will report some
examples already known, for illustrative purposes. We then try to connect the
Bayesian approach with the NNPDF fits based on replicas, which contain the
results of the multple replicas. We can anticipate here that the NNPDF procedure
for Gaussian data is able to characterise completely the posterior proabillity
density. We then use the Bayesian formalism to try understand how to quantify systematic errors by performing a set of self consistent firs. 






{\bf to be written. Keeping the text below for reference}

In an NNPDF fit to experimental data, a set of replicas are fitted to pseudodata
which is generated according the experimental central values and uncertainties.
A successful fit is realised if the resulting replicas agree with the underlying
law within uncertainties. Since the underlying law is usually unknown we cannot
directly measure if this is the case.

We can, however, test the methodology in a fit to artificial data, which is
generated from theory predictions from an input PDF and check that we agree with
the input PDF within uncertainties. This is the basis of a closure test, which
has already been used to test to validity of previous iterations of the NNPDF
methodology. Here we aim to refine some of the pre-existing closure test
estimators and with the help of fast fitting methodology perform a more
extensive study of how faithful our uncertainties are.
