\section{Introduction}
\label{sec:Intro}

{\bf to be written. Keeping the text below for reference}

In an NNPDF fit to experimental data, a set of replicas are fitted to pseudodata
which is generated according the experimental central values and uncertainties.
A successful fit is realised if the resulting replicas agree with the underlying
law within uncertainties. Since the underlying law is usually unknown we cannot
directly measure if this is the case.

We can, however, test the methodology in a fit to artificial data, which is
generated from theory predictions from an input PDF and check that we agree with
the input PDF within uncertainties. This is the basis of a closure test, which
has already been used to test to validity of previous iterations of the NNPDF
methodology. Here we aim to refine some of the pre-existing closure test
estimators and with the help of fast fitting methodology perform a more
extensive study of how faithful our uncertainties are.
