\section{Gaussian integrals}
\label{sec:GaussianIntegrals}

Theory errors can be included in this framework by allowing the distribution of
observables around the theory prediction to have a finite width, \eg\ by
replacing the Dirac delta 
\begin{equation}
    \label{eq:DiracInApp}
    \delta(y-\mathcal{G}u)
\end{equation}
in Eq.~\ref{xxx} with a Gaussian 
\begin{equation}
    \label{eq:TheoryGaussian}
    \theta(\obs,\modelvec|\fwdobsop) \propto \exp\left[
        -\frac12 \left(y-\mathcal{G}u\right)^T
        C_T^{-1} \left(y-\mathcal{G}u\right)
    \right]\, .
\end{equation}
For the purposes of this study, we do not want to provide a realistic estimate
of theory errors. Instead we will be assuming that the errors are uncorrelated
and identical for all data points
\begin{equation}
    \label{eq:DiagTheoryCov}
    C_T = \sigma^2 \mathds{1}\, ,
\end{equation}
and we will be interested in the limit where $\sigma^2\to 0$. 

\subsection{Integrating out the data}
\label{sec:IntOutData}

Marginalizing with respect to \obs\ in this case yields 
\begin{align}
  \label{eq:MarginGaussData}
  \pi_M(\modelvec|\obspriorcent,\modelpriorcent,\fwdobsop) 
  &\propto \pi_{M}^0(\modelvec|\modelpriorcent) \, 
  \int dy\, \pi_{D}^0(\obs|\obspriorcent) 
    \theta(\obs,\modelvec|\fwdobsop) \, .
\end{align}
The argument of the exponential in the integrand is a quadratic form in \obs, 
\begin{equation}
    \label{eq:QuadFormDataInt}
    A = \left(y-y_0\right)^T C_D^{-1} \left(y-y_0\right) +
    \left(y-\mathcal{G}u\right)^T C_T^{-1} \left(y-\mathcal{G}u\right)\, .
\end{equation}
The integral can be easily evaluated by completing the square, 
\begin{equation}
    \label{eq:QuadFormDataIntSquare}
    A = \left(y-\tilde{y}\right)^T 
    \tilde{C}_D^{-1}
    \left(y-\tilde{y}\right) + R_D\, .
\end{equation}
Comparing Eqs.~\ref{eq:QuadFormDataInt} and~\ref{eq:QuadFormDataIntSquare} at order $y^2$ and \obs, yields
\begin{align}
    \tilde{C}_D^{-1} &= \frac{1}{\sigma^2}
    \left(\mathds{1} + \sigma^2 C_D^{-1}\right)\, , \\
    \tilde{y} &= \left(\mathds{1} + \sigma^2 C_D^{-1}\right)^{-1} 
    \left(
        \mathcal{G}u + \sigma^2 C_D^{-1} y_0
    \right)\, ,
\end{align}
and therefore
\begin{align}
    \tilde{y}^T \tilde{C}_D^{-1} \tilde{y}
    &= \frac{1}{\sigma^2} \left(\mathcal{G}u\right)^T
    \left(\mathds{1}+\sigma^2 C_D^{-1}\right)^{-1} \left(\mathcal{G}u\right) +
    y_0^T C_D^{-1} \left(\mathds{1}+\sigma^2 C_D^{-1}\right)^{-1} 
    \left(\mathcal{G}u\right) + \nonumber \\
    \label{eq:RemainderFromSquare}
    & \quad + \left(\mathcal{G}u\right)^T C_D^{-1} 
    \left(\mathds{1}+\sigma^2 C_D^{-1}\right)^{-1} y_0 + 
    \sigma^2 y_0^T C_D^{-1} \left(\mathds{1}+\sigma^2 C_D^{-1}\right)^{-1} 
    C_D^{-1} y_0\, .
\end{align}
Note that the four terms in the equation above are ordered in increasing powers
of $\sigma^2$ and ultimately we will be interested in the limit $\sigma^2\to 0$,
which reproduces the Dirac delta in $\theta(y,u)$. Plugging
Eq.~\ref{eq:RemainderFromSquare} in Eq.~\ref{eq:QuadFormDataIntSquare} and again
comparing to Eq.~\ref{eq:QuadFormDataInt}, we find
\begin{align}
    \label{eq:RDBeforeLimit}   
    R_D 
    &= \frac{1}{\sigma^2} \left(\mathcal{G}u\right)^T 
    \left[
        1 - \frac{1}{1+\sigma^2 C_D^{-1}} 
    \right]
    \left(\mathcal{G}u\right) - y_0^T C_D^{-1} \left(\mathcal{G}u\right)
    - \left(\mathcal{G}u\right)^T C_D^{-1} y_0 + y_0^T C_D^{-1} y_0\, ,
\end{align}
Expanding and keeping only finite terms for $\sigma^2 \to 0$ we finally obtain
\begin{equation}
    \label{eq:RDAfterLimit}
    R_D = \left(\mathcal{G}u - y_0\right)^T C_D^{-1}
    \left(\mathcal{G}u - y_0\right)\, .
\end{equation}



