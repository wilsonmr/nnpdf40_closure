\documentclass[11pt,a4paper]{article}

\usepackage[colorlinks=true, linkcolor=black!50!blue, urlcolor=blue, citecolor=blue, anchorcolor=blue]{hyperref}
\usepackage[font=small,labelfont=bf,margin=0mm,labelsep=period,tableposition=top]{caption}
\usepackage[a4paper,top=3cm,bottom=2.5cm,left=2.5cm,right=2.5cm,bindingoffset=0mm]{geometry}

\usepackage{graphicx}
\usepackage{float}
\usepackage{afterpage}
\usepackage{epsfig,cite}
\usepackage{amssymb}
\usepackage{amsmath}
%\usepackage{dsfont}
\usepackage{multirow}
\usepackage{url}
\usepackage{xcolor}
\usepackage{float}
\usepackage{afterpage}


\usepackage{url}

\usepackage{booktabs}
\usepackage{tikz}
\usetikzlibrary{arrows}
\usepackage{tikz-3dplot}
\usepackage{enumitem}
\usepackage{hyperref}
\usepackage{cite}
\usepackage{simpler-wick}
\setlength{\parindent}{0pt}
\graphicspath{{./figs/}}

%%%%%%%%%%%%%%%%%%%%%%%%%%%%%%%%%%%%%%%%%%%%%%%%%%%%%%%%%%%%%

\def\smallfrac#1#2{\hbox{$\frac{#1}{#2}$}}
\newcommand{\lp}{\left(}
\newcommand{\rp}{\right)}
\newcommand{\bare}{{(0)}}
\newcommand{\sym}{\mathrm{sym}}
\newcommand{\asy}{\mathrm{asy}}
\newcommand{\as}{\alpha_s}
\newcommand{\comment}[1]{\textbf{\textcolor{red!60!black}{[#1]}}}
\newcommand{\unsure}[1]{\textbf{\textcolor{red!60!black}{[#1]}}}
\newcommand{\nsv}{\mathrm{V}_3}
\newcommand{\nst}{\mathrm{T}_3}
\newcommand{\eg}{{\em e.g.}}
\newcommand{\ie}{{\em i.e.}}
\newcommand{\msbar}{\overline{MS}}
\newcommand{\dis}{\text{DIS}}
\newcommand{\pos}{\text{POS}}

\def \z{\zeta}

\numberwithin{equation}{section}
\numberwithin{figure}{section}
\numberwithin{table}{section}


\usepackage{tabularx}
\newcolumntype{C}[1]{>{\centering\arraybackslash}p{#1}}

\begin{document}
We thank the referee for their close reading of our paper and for the detailed comments and criticisms 
which helped us clarifying different points.
\begin{enumerate}
    \item \textcolor{red}{TG: which was the reference to Hadamar?}
    \item We have added a sentence at the end of the paragraph to stress this points.
    \item As far as we are aware, this kind of demonstration has never been explicitly
    shown in the literature before however we agree that much of the analysis on
    MC replicas relies on this property of the replicas. We believe that this explicit example
    is instructive to connect to intuitive understanding of the MC approach and
    the Bayesian framework discussed in the paper. Based on the referee's comment
    we have updated to language at the start of this section to reflect this.
    \item We have added the explicit definition of the average over test and training data and 
    clarify the statement that ``the estimators are independent of the test data''. Indeed with dependence
    on training and test data we meant the explicit or implicit dependence on the parameters $\eta$ and $\eta'$.
    \item We have added a sentence to clarify that by single replica proxy fits we mean fits
    run over different training sets made by a single replica. \textcolor{red}{Is it true? Is it what done in NNPDF30
    paper?}
    \item In the example of Sec4.2 training and set tests are indeed taken to be the same for simplicity.
    We have added a sentence stating explicitly thi choice.
    \item This is a typo, $y_0$ should not be there. We have remove it, and clarified the sentence following Eq.(124),
    which in the revised version of the paper is Eq.(125).\textcolor{red}{TG: correct?}
    \item In order to answer the question properly one should repeat the analysis using the 
    NNPDF3.1 code and look at the estimators to see if the NNPDF3.1 uncertainties are indeed faithful according to 
    the criteria presented here.
    The computational cost of this has not been judged to be feasible at the moment.
    The point we would like to make with this work is to give
    a better definition of what we mean by faithful uncertainties and passing a closure test.
    In our opinion this is filling a gap in the previous NNPDF literature where these points 
    were not fully defined and addressed, and we hope that this work gives more solid criteria 
    to test the methodologies of future releases, starting from the current NNPDF4.0 methodology.
    %
    Performing the full exercise on the old methodology to assess whether or not it passes the faithful
    criteria defined here, despite interesting, is not meant to be part of this work.
    However as noticed by the referee we can add that these estimators cannot distinguish 
    between two faithful methodology, independently of the relative size of their uncertainties.
    Therefore, supposing the NNPDF3.1 are faithful (which has not been tested explicitly),
    we should expect to get similar values to those we got for NNPDF4.0.
    We have added a paragraph to discuss these points, using also the example of Fig.3 suggested by the referee,
    that we found very clear to understand qualitatively how faithfulness and size of the uncertainties are 
    not necessarily related. 

    \textcolor{red}{TG: can we say something more using some old NNPDF3.0 estimators?}
    \item Better methodologies should indeed aim to reduce the L0 and L1 uncertainties.
    However, a feature that a good methodology should have is that these components of the total uncertainty should
    still be the dominant ones in the regions where data are not present.
    Therefore we do want to have a bigger L0 and L1 uncertainties at small and large-x.
    However this should be driven by the lack of experimental information in these kinematic regions,
    and not by some unefficiency of the fitting methodology. What we meant here is that better methodologies 
    should reduce the size of these uncertainties as far as the unefficiency of the fitting procedure
    is concerned. We suggest that this is what we might be observing in the comparison
    between NNPDF4.0 and NNPDF3.1 of Fig.5, and in support of this we note that L0 and L1 uncertainty
    are reduced in both the data and extrapolation region, while still remaining the dominant source of 
    error in the latter case.    
    However the question concerning how big these components should be and how we could validate them 
    in a closure test is not answered in this paper, since the estimators described here 
    are computed on the actual data.
    This is an issue which has still to be studied, and will be addressed in future works.
    We have added few lines to elaborate on this. Further work in this direction is required to 
    asses whether or not we are already reaching a fundamental limit for L0 and L1.
    \item We thank the referee for bringing Fig. 8 to our attention, indeed the x-axis was
    not explained properly in the caption. We have updated to caption to try and clarify what is being shown.
    As the referee suspects we do indeed expect the distributions to follow the same shape, and we believe
    any discrepancies here can be explained by the bias distribution (green histogram)
    having relatively low statistics, as such we only use this as qualitative evidence
    that the uncertainities are unfaithful, but rely on Table 1 - where we just compare the means of these
    distributions for a quantitative comparison.
    \item We agree with the referee and we have added few sentences to stress the point.
    \item We agree with the point of the referee, and we have modified the sentence accordingly.
\end{enumerate}





\bibliographystyle{UTPstyle}
\bibliography{main}
\end{document}
