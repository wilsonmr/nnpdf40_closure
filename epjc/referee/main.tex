\documentclass[11pt,a4paper]{article}

\usepackage[colorlinks=true, linkcolor=black!50!blue, urlcolor=blue, citecolor=blue, anchorcolor=blue]{hyperref}
\usepackage[font=small,labelfont=bf,margin=0mm,labelsep=period,tableposition=top]{caption}
\usepackage[a4paper,top=3cm,bottom=2.5cm,left=2.5cm,right=2.5cm,bindingoffset=0mm]{geometry}

\usepackage{graphicx}
\usepackage{float}
\usepackage{afterpage}
\usepackage{epsfig,cite}
\usepackage{amssymb}
\usepackage{amsmath}
%\usepackage{dsfont}
\usepackage{multirow}
\usepackage{url}
\usepackage{xcolor}
\usepackage{float}
\usepackage{afterpage}


\usepackage{url}

\usepackage{booktabs}
\usepackage{tikz}
\usetikzlibrary{arrows}
\usepackage{tikz-3dplot}
\usepackage{enumitem}
\usepackage{hyperref}
\usepackage{cite}
\usepackage{simpler-wick}
\setlength{\parindent}{0pt}
\graphicspath{{./figs/}}

%%%%%%%%%%%%%%%%%%%%%%%%%%%%%%%%%%%%%%%%%%%%%%%%%%%%%%%%%%%%%

\def\smallfrac#1#2{\hbox{$\frac{#1}{#2}$}}
\newcommand{\lp}{\left(}
\newcommand{\rp}{\right)}
\newcommand{\bare}{{(0)}}
\newcommand{\sym}{\mathrm{sym}}
\newcommand{\asy}{\mathrm{asy}}
\newcommand{\as}{\alpha_s}
\newcommand{\comment}[1]{\textbf{\textcolor{red!60!black}{[#1]}}}
\newcommand{\unsure}[1]{\textbf{\textcolor{red!60!black}{[#1]}}}
\newcommand{\nsv}{\mathrm{V}_3}
\newcommand{\nst}{\mathrm{T}_3}
\newcommand{\eg}{{\em e.g.}}
\newcommand{\ie}{{\em i.e.}}
\newcommand{\msbar}{\overline{MS}}
\newcommand{\dis}{\text{DIS}}
\newcommand{\pos}{\text{POS}}

\def \z{\zeta}

\numberwithin{equation}{section}
\numberwithin{figure}{section}
\numberwithin{table}{section}


\usepackage{tabularx}
\newcolumntype{C}[1]{>{\centering\arraybackslash}p{#1}}

\begin{document}
We thank the referee for their close reading of our paper and for the detailed comments and criticisms 
which helped us clarifying different points.
\begin{enumerate}
    \item \textcolor{red}{TG: which was the reference to Hadamar?}
    \item We have added a sentence at the end of the paragraph to stress this points.
    \item This sort of demonstration had indeed been stated in words in previous NNPDF literature,
    for example in Sec.4.1 of \cite{Forte:2002fg} in the context of structure function fits:
    at the end of the third paragraph it is stated that it is easy to see that
    the model has the same average, variance and point-to-point correlation
    as the artificial replica distribution. In this paragraph of our paper
    we are just recalling this point to the reader for completeness, spelling out all the equations for clarity.
    Also, we wanted to stress the fact that this demonstration only strictly applies for the linear case,
    while in the case of hadronic observables we also rely on a linear approximation of the 
    forward map. We have added a citation to Ref.~\cite{Forte:2002fg}, to clarify that this point
    had been already addressed in the past.
    \item we have added the explicit definition of the average over test and training data and 
    clarify the statement that the estimators are independent of the test data. Indeed with dependence
    on training and test data we meant dependence on the parameters $\eta$ and $\eta'$.
\end{enumerate}





\bibliographystyle{UTPstyle}
\bibliography{main}
\end{document}
