\section{Summary}

We have presented a theoretical framework for treating inverse problems from a
Bayesian perspective. In particular, the framework provides a more formal
description of what it means when we talk about propagating experimental
uncertainties into the space of models by focusing on the posterior probability
measure in model space given the available data. Strictly speaking, there is no
fitting required to obtain the expression for the posterior distributions in the
space of the data or the model, instead these are obtained by marginalising the
joint distribution. We note that sampling from the posterior distribution of the
model is, in general, highly non-trivial; however we can show that at least for
linear problems the NNPDF MC methodology produces a sample of models which are
distributed according to the posterior model distribution predicted from the
Bayesian formalism. Furthermore, we provide evidence that even for non-linear
models this result at least holds as a good approximation close to the MAP
estimator.

We then use this formal framework to think about some of the estimators which we
use as part of the NNPDF closure test. In particular we derive bias and variance
from decomposing an out of sample error function, which is understood from a
classic fitting point of view. The estimators are then related back to the
posterior distributions in the Bayesian framework. We note that the estimators
themselves are not perfect and suffer from only testing the model uncertainties
locally (in regions where the test data probes). Furthermore, the estimators
only give an approximate overall picture, and cannot be used to diagnose where
the problem arises if we find evidence that the model uncertaintes are not
faithful.

Given the framework set out here, future work should be undertaken to generalise
the closure estimators to model space. This would likely involve a combination
of the closure estimators presented here and the extension of the Bayesian
framework to infinite-dimensional spaces.

We give some closure results, as a proof of principle. The results
presented here are examined in more detail alongside the full NNPDF4.0
release \cite{NNPDF40} but serve as an example of how the data space
estimators can be practically included even in a rather complex setting. 
The NNPDF4.0 methodology
passes the closure test according to the estimators that we analysed, providing
evidence that for unseen data the current NNPDF methodology appears to provide
faithful uncertainties. The estimators are not limited to this specific
application, and the results presented demonstrate how the data space estimators
can be incorporated into any inverse problem. As previously mentioned, a more
general set of estimators in model space would be the gold-standard and give us
confidence in our uncertainties for future observables which probe regions which
are not covered by either the training or testing data.

We stress that, if the ultimate goal of closure tests is 
to validate whether the uncertainty of a given methodology is faithful
in the context of a global fit, this is not fully achieved by the estimators presented here:
in the closure test framework described in this study, the prior on the data is
fully consistent with the generation of the observable central values from the
true observables and uncertainties by construction, which is often not the case
in real world fits. Something which has to be investigated is how we can
guarantee faithful uncertainties when this is no longer the case. For example
one could consider a closure test where the generated data was inconsistent with
the prior. The closure tests presented here should therefore be seen as the first necessary step 
in this direction, and future study will aim to complete the picture.

Some of the observed inconsistentency between the prior and the data central
values might be due to missing theoretical uncertainties, but once all sources
of theoretical uncertainties have been accounted for there could still be
tension in the data. The advantage of viewing inverse problems from a Bayesian
perspective is access to methodologies which deal with inconsistent data (or
unknown systematics) in a Bayesian framework, for example in these cosmological
studies~\cite{Luis_Bernal_2018,Hobson_2002}. These methods potentially offer a
more formal approach to dealing with inconsistent data, rather than ad hoc
procedures.

The inclusion of MHOUs in the likelihood was justified from a Bayesian
perspective~\cite{AbdulKhalek:2019ihb}, but here we have drawn a connection
between the model replicas and the posterior distribution in model space, which
explains why theoretical uncertainties must be included in both the sampling of
the pseudodata replicas and the likelihood. Therefore we emphasise that the
framework set out here is not only useful for understanding model uncertainties
with the current methodology, but also for motivating future methodological
development from a Bayesian perspective. We defer more detailed investigations
of the potential of this approach to future work. 

\section*{Acknowledgements}
We thank the members of the NNPDF collaboration for discussions and comments on the present work. 
L.~D.~D. is supported by the U.K. Science and Technology Facility Council
(STFC) grant ST/P000630/1, T.~G. is supported by NWO via a ENW-KLEIN-2 project and
M.~W. by the STFC grant ST/R504737/1.
