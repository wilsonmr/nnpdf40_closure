\section{Results}

\subsection{Bias-variance ratio $\biasvarratio$}

We calculated $\biasvarratio$ on the test data, shown in
Tab.~\ref{tab:summarise_new_data}. An
uncertainty on $\biasvarratio$ by performing a bootstrap sample
\cite{efron1994introduction},
where we randomly sample from both fits and replicas and re-calculate
$\biasvarratio$, the value and error presented in the table is then the mean
and standard deviation across bootstrap samples. We checked that the distribution
of the estimator across bootstrap samples is indeed Gaussian.

\begin{table}
    \begin{center}
        \begin{tabular}{lr}
            \toprule
            {} &  $\biasvarratio$ \\
            experiment &                      \\
            \midrule
            ATLAS       &                 $1.04 \pm 0.04$  \\
            CMS         &                 $1.04 \pm 0.06$ \\
            LHCb       &                 $0.82 \pm 0.06$ \\
            Total       &                 $ 1.03 \pm 0.05$ \\
            \bottomrule
        \end{tabular}
    \end{center}
    \caption{
        The bias-variance ratio, $\biasvarratio$, for unfitted data, summarised in
        Tab.~\ref{tab:summarise_new_data}. Generally $\biasvarratio$ is within
        $1\sigma$ of 1, except for LHCb where there is a $3\sigma$ deviation.
        There are, however fewer datapoints included in LHCb and so this could be
        a statistical fluctuation. Furthermore, if this result were to hold as
        more data was added in these kinematic regions, it suggests that the
        uncertainties are slightly overestimated (on the level of 20\%) which is
        is at worst a bit conservative, and preferable to underestimated
        uncertainities. Finally, across all data we find $\biasvarratio$ to
        be compatible with 1.
    }
    \label{tab:biasvarratio}
\end{table}

One can also compare qualitatively the distribution of bias across fits, to the
distribution of the difference between replica predictions and expectation
values of predictions (in units of the covariance) across different fits
and replicas. The square root ratio of the mean of these two distributions
is precisely $\biasvarratio$.

\begin{figure}
    \centering
    \includegraphics[width=0.6 \textwidth]{plot_bias_variance_distributions_total.png}
    \caption{The green histogram is the distribution of the total bias across fits,
    the orange histogram is the distribution of the difference between the
    replica and central predictions squared, in units of the covariance
    across all fits and replicas. This gives a qualitative picture of the full
    distribution, in Tab.~\ref{tab:biasvarratio} we compare the square root of the
    mean of each distribution.}
\end{figure}

\subsection{Comparison to $\xi_{1\sigma}$}

As discussed in Sec.~\ref{sec:QuantileStatistics}, one can define an analogous
estimator in data space, based upon $\xi_{n\sigma}$, which was defined on a grid
of points in $x$ and $Q^2$ in PDF space in \cite{nnpdf30}. There is not
a one-to-one correspondence
between this and $\biasvarratio$, but a loose approximation using
Eq.~\ref{eq:expectedxi}. In Tab.~\ref{tab:xicomparison} we compare the estimated
$\xi_{1\sigma}$ from
subsituting $\biasvarratio$ into Eq.~\ref{eq:expectedxi} and to the
measured value.

\begin{table}
    \begin{center}
        \begin{tabular}{lrr}
            \toprule
            {} &  $\xi_{1\sigma}$ &  est. from $\biasvarratio$ \\
            experiment &  &                \\
            \midrule
            ATLAS       &  $0.70 \pm 0.02$ &  $0.66 \pm 0.02$ \\
            CMS         &  $0.68 \pm 0.02$ &  $0.67 \pm 0.03$ \\
            LHCb        &  $0.69 \pm 0.04$ &  $0.78 \pm 0.04$ \\
            Total       &  $0.69 \pm 0.02$ &  $0.67 \pm 0.02$ \\
            \bottomrule
        \end{tabular}
    \end{center}
    \caption{
        Comparing the measured value of $\xi_1\sigma$ and the estimated
        value from $\biasvarratio$. The two columns are consistent, which
        suggests the approximation that the ratio of uncertainties is
        approximately the same across all data is not completely invalidated.
        The largest discrepancy is with LHCb, which in Tab.~\ref{tab:biasvarratio}
        was shown to possibly have slightly overestimated uncertainties. Here we
        see that the measure $\xi_1\sigma$ is completely consistent with 0.68,
        which possibly suggests that the $\biasvarratio$ is dominated by some
        eigenvectors of the uncertainty having. The breakdown of $\xi_1\sigma$
        for each eigenvector of LHCb can be seen in
        Fig.~\ref{fig:xibreakdownlhcb}, which further supports this.
    }
    \label{tab:xicomparison}
\end{table}

Despite the assumptions entering each of the two estimators differing, we see
good agreement between the $\xi_{1\sigma}$ estimated from $\biasvarratio$
and that measured directly. We find this result reassuring, since it indicates
not only that the total uncertainty averaged across all data is faithful, but
also that the uncertainty on each data point seems faithful. If the results
differed it would indicate some kind of imbalance, where some components
of the uncertainty are correctly represented by the replicas but other directions
are not.

\begin{figure}
    \centering
    \includegraphics[width=0.6 \textwidth]{lhcbxi_data.png}
    \caption{$\xi_1\sigma$ for each eigenvector of the experimental
    covariance matrix. We see that the first few eigenvectors have over
    estimated uncertainty, but the latter are more evenly distributed
    around 0.68. The suggested overestimated uncertainty by $\biasvarratio$
    is likely dominated by the first few eigenvectors.}
    \label{fig:xibreakdownlhcb}
\end{figure}
