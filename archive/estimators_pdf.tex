\section{Calculating estimators in PDF space}

{\bf THIS SHOULD EITHER GO OR GO IN APPENDIX}

We can calculate the same set of estimators in PDF space with minor alteration
to the definitions. Instead of calculating the estimators on points in data space,
we instead choose to sample the PDFs at the initial scale at discrete intervals
in $x$ and flavour. For this study the
points in x for singlet and gluon were chosen to be half logarithmically spaced
between $10^-3<x<0.1$ and half linearly spaced between $0.1<x<0.5$. For the
other flavours (V, V3, V8, T3, and T8) we simply choose the points to be
linearly spaced $0.1<x<0.5$ which is supposed to roughly capture the data region
of the PDFs.

The covariance matrix which is used in the definition of bias and variance is
estimated from the union of all of the replicas, allowing for correlations
across different flavours. We can also calculate $\xi_{1\sigma}$ in the basis
which diagonalises this covariance matrix and examine the
agreement between the expected $\xi_{1\sigma}$ from calculated from
$\frac{\eshift{\bias}}{\eshift{\var}}$, using eq. \eqref{eq:expectedxi} and
the value measured directly from the PDFs. Clearly there is a slight circularity
when using this definition of the covariance matrix
\begin{equation}
    \Sigma_{ij} = \frac{1}{\nfits}\frac{1}{\nreps} \sum_{l}^{\nfits} \sum_{k}^{\nreps} \diffcentrep^{(l)}_i \diffcentrep^{(l)}_j ,
\end{equation}
where $\Sigma_{ij}$ is the PDF space covariance matrix, so indices $i$ and $j$
run over all points in $x$ and flavour, and $\model$ now refers to a point in
PDF space. Because if our assumption is correct, that the distribution of
replicas around the central value is the same for all fits in the limit of infinite
replicas, then in this limit the variance for a given fit should tend to
exactly 1. The definition of expected bias then becomes
\begin{equation}
    \begin{split}
        \eshift{\bias} &= \frac{1}{\nfits} \frac{1}{\ndata} \sum^{\nfits}_{l} \sum^{\ndata}_{ij}
    \diffcentunder^{(l)}_{i} \Sigma^{-1}_{ij} \diffcentunder^{(l)}_{j} \\
    &= \frac{1}{\ndata} \sum_{ij}^{\ndata} \mathcal{C}_{ij} \Sigma^{-1}_{ij} \\
    &= \frac{1}{\ndata} {\rm Tr}[\mathcal{C}\Sigma^{-1}]
    \end{split}
\end{equation}
where we have labelled $\mathcal{C}$ the covariance of the central PDFs around
the underlying law and denoted the number of points in flavour and $x$ as
$\npoints = N_{x} \times N_{\rm flavours}$. We see then that requiring that
$\frac{\eshift{\bias}}{\eshift{\var}}$
when the covariance is estimated from the replicas themselves is actually
a more stringent check on whether the second moment of the distribution of
replicas around the central PDF is equal to the second moment of the distribution
of central PDFs around the underlying law. Since it accounts for all correlations
in each of the distribution, instead of just being a ratio of the mean squared
radius.
It is theoretically possible
to calculate the analogue in data space, however in practice the number of points
chosen in PDF space is $~10$ and so the covariance matrix can be estimated quite
well. In data space, the number of points is $~1000$ and so the minimum number
of replicas estimate a positive definite matrix is far greater, that also doesn't
guarantee a stable estimation of the covariance, in the case that correlations
between points is high then a much large sample is required to get a reliable
estimate of the covariance matrix.
