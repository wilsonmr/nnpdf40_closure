\section{Estimators}
Since we have access to the underlying predictions in a closure test we can
define some new estimators which can be used to determine how well a fit performs.

We start with the expectation of $\repchis$ across replicas
\begin{equation}
    \erep{\repchis} = \frac{1}{\ndata}\erep{ \sum_{ij} \diffreptwo_i \invcov{ij} \diffreptwo_j}.
\end{equation}
we can perform a decompositon of the above, by completing the square into
4 terms
\begin{equation}
    \begin{split}
        \erep{\repchis} &= {\rm noise} + {\rm bias} + {\rm variance} - {\rm cross\,term} \\
        &=  {\rm noise} + {\rm variance} + \Delta_{\chi^2}, \\
        &= \chi^2 + {\rm variance}
    \end{split}
\end{equation}
where $\Delta_{\chi^2}$ was an estimator which has been discussed in previous
NNPDF closure test studies, and $\chi^2$ was defined in eq. \eqref{eq:centralchi2}.

\subsection{Bias}

Defined as the difference between the central value of the replica predictions
and the underlying predictions in units of the covariance, given by
\begin{equation}
    \bias = \frac{1}{\ndata} \sum_{ij} \diffcentunder_i \invcov{ij} \diffcentunder_j.
\end{equation}
It is desirable for a fit to have a smaller bias because that indicates
that the fit is reproducing the underlying predictions well.

\subsection{Variance}

The variance of the replica predictions in units of the covariance
\begin{equation}
    \var = \frac{1}{\ndata} \erep{ \sum_{ij} \diffcentrep_{i} \invcov{ij} \diffcentrep_{j}},
\end{equation}
which can be interpreted as the uncertainty of the PDF in the space of data. It's
important to note that both variance and bias can be determined purely from the
ensemble of PDF replicas and the underlying law.

\subsection{Multiple Closure Fits}

As well as being able to calculate the bias estimator in a closure test and
see how well the underlying law is reproduced, an advantage to testing the
methodology with closure tests is the ability to generate ensembles of
level one data - or experimental central values. Running multiple closure fits
on different configurations of level one shifts facilitates testing
whether or not the distribution of replica predictions about the mean of the
predictions for a given closure fit is statistically representative of the
distribution of central value of the predictions for different closure fits
about the underlying law. If we find that
the two distributions are indeed the same then this supports the case that
the methodology works given some caveats - such as the distribution of central
values being accurately modelled by a multigaussian distribution with
covariance given by the uncertainties which are provided by experimentalists.

There are different estimators to test if this
is the case each with disadvantages and advantages.

\subsubsection{Bias-Variance Ratio}

If the uncertainty is faithful then we would expect that upon refitting the
difference between the central value of the replica predictions and the underlying
law is represented properly by the uncertainty of a given PDF. If this is true
then if we were to perform multiple fits with different shifts and calculate the
expectation value of the bias across fits, $\eshift{\bias}$, then we should find
that this equals $\eshift{\var}$, or equivalently
\begin{equation}
    \frac{\eshift{\bias}}{\eshift{\var}} = 1.
\end{equation}
We note that this quantity is slightly coarse: we are checking that the mean square
difference between central predictions and underlying law is the same as the
mean square difference between replica predictions and their central values.
It's also worth noting that this quantity is a squared quantity, the square
root of this can be interpreted as how much uncertainty has been over or
underestimated e.g $\sqrt{\frac{\eshift{\bias}}{\eshift{\var}}} = 0.5$ would
mean the uncertainty for a given fit is, on average, over estimated by a factor
of 2.

\subsubsection{Quantile statistics}

A closure test estimator which was previously defined in PDF space was $\xi_{1\sigma}$.
We define here an analogous estimator in data space
\begin{equation}
    \xi_{1\sigma} = \frac{1}{\ndata} \frac{1}{\nfits} \sum_{i}^{\ndata} \sum_{l}^{\nfits}
    I_{[-\sigma_i^{(l)}, \sigma_i^{(l)}]}
    \left( \erep{\model_i}^{(l)} - \law_i \right),
\end{equation}
where $\sigma^{(l)}$ is the standard deviation of the theory predictions
estimated from the replicas of fit $l$. $\xi_{1\sigma}$ aims to measure the same
thing as $\frac{\eshift{\bias}}{\eshift{\var}}$: whether the distribution of
replicas for a given fit matches the distribution of the central predictions
around the underlying predictions. It's useful to define $\xi_{1\sigma}^{i}$ as
the value of $\xi_{1\sigma}$ for an individual data point
such that
\begin{equation}
    \xi_{1\sigma} = \frac{1}{\ndata} \sum_i \xi_{1\sigma}^{i}.
\end{equation}
If we assume that
the replica distribution is constant across
fits then our definition $\xi_{1\sigma}^{i}$ is just a discretised expected value
of the indicator function. In the case that we have infinite fits, then
\begin{equation}
    \xi_{1\sigma}^{i} = \int_{-\infty}^{\infty} I_{[-\sigma_i, \sigma_i]}
    p(\erep{\model_i} - \law_i)
    {\rm d}(\erep{\model_i} - \law_i).
\end{equation}
If we then assume that $p(\erep{\model_i} - \law_i)$ is a gaussian centered on zero
with a standard deviation which we will denote as $\modelstd_i$ then the integral
simplifies to
\begin{equation}
    \xi_{1\sigma}^{i} = \erf \left( \frac{\sigma_i}{\modelstd_i \sqrt{2}}\right),
    \label{eq:expectedxi}
\end{equation}
which is the standard result of integrating a gaussian over some finite symmetric
interval. Clearly if the distribution of central predictions about the underlying law
matches the distribution of the replica predicitons around the central predictions
then the expected value of $\xi_{1\sigma}^{i}$ is 0.68. This is consistent with
the assumptions we made, it's the quantile statistics of a gaussian distribution.

One can also look at the variance of the indicator function across fits to
get an idea of the fluctuation of $\xi_{1\sigma}^{i}$, which we will denote
as $\Delta[\xi_{1\sigma}^{i}]$
\begin{equation}
    \begin{split}
        \Delta[\xi_{1\sigma}^{i}] =& \int_{-\infty}^{\infty} I_{[-\sigma_i, \sigma_i]}^2
        p(\erep{\model_i} - \law_i)
    {\rm d}(\erep{\model_i} - \law_i) - \\
    &\left( \int_{-\infty}^{\infty} I_{[-\sigma_i, \sigma_i]}
    p(\erep{\model_i} - \law_i)
    {\rm d}(\erep{\model_i} - \law_i) \right)^2,
    \end{split}
\end{equation}
which can be simplified to
\begin{equation}
    \Delta[\xi_{1\sigma}^{i}] =
    \erf \left( \frac{\sigma_i}{\modelstd_i \sqrt{2}}\right) -
    \erf \left( \frac{\sigma_i}{\modelstd_i \sqrt{2}}\right)^2.
\end{equation}
Even in the ideal case that the distributions are the same, we see that the
$\Delta[\xi_{1\sigma}^{i}] = 0.22$, which is a large spread considering
$\xi_{1\sigma}^{i}$ is bounded between 0 and 1.

When taking the mean across datapoints to obtain $\xi_{1\sigma}$ we note that
getting 0.68 relies on each sampled $\xi_{1\sigma}^{i}$ being statistically
independent. This clearly will not be the case because there can be a big correlation
between datapoints within the same dataset. We can calculate $\xi_{1\sigma}$
in different basis and note that unlike $\chi^2$ and quantities of the form
$v^T M v$, $\xi_{1\sigma}$ is
not basis independent. There is a choice of basis, however it will be useful to
compare the value $\xi_{1\sigma}$ and $\frac{\eshift{\bias}}{\eshift{\var}}$.
Therefore, a natural basis to calculate
$\xi_{1\sigma}$ is the basis which diagonalises the experimental covariance matrix
because bias and variance are both calculated in units of the experimental
% Should I mention this without having tried it?
covariance. Alternatively one could estimate the covariance of the difference
between replicas and central values and then rotate the differences between
the central predictions and the underlying law into the basis which diagonalises
this replica covariance matrix.

\subsection{Out of sample}

We want to calculate the closure test estimators on data which is out of sample
to check uncertainty is faithful on predictions.

\paragraph{TODO}{Review the toy model here and show that in/out of sample
didn't make difference}
