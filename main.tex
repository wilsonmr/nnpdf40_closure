\documentclass[11pt]{article}

%\pagestyle{headings}

\textwidth=440pt
\hoffset=-0.6truein

\usepackage{amsmath}
\usepackage{amsfonts}
\usepackage{amssymb}
\usepackage{dsfont}
\usepackage{pifont}
\usepackage{booktabs}
%\usepackage{bbold}
\usepackage{graphicx}
\usepackage{epstopdf}
\usepackage{epsfig}
%\usepackage{bibunits}
%\usepackage{theorem}
\usepackage[framed]{ntheorem}
\usepackage{framed}
%\usepackage{showlabels}
\usepackage{makeidx}
\usepackage{simplewick}
%\usepackage{tikz-feynman}
\usepackage{hyperref}
\usepackage{placeins}
\usepackage[font=small,labelfont=bf]{caption}
%\tikzfeynmanset{compat=1.0.0}

% data levels
\newcommand{\levone}{z}
\newcommand{\levtwo}{y}
\newcommand{\law}{f}
\newcommand{\model}{g}
\newcommand{\shift}{\eta}
\newcommand{\noise}{\epsilon}
\newcommand{\lawmodel}{w}

% define vector
\newcommand{\vv}[1]{\mathbf{#1}}
% should we add a fit index?
\newcommand{\vecdiffreptwo}{\left( \vv{\model}^{\repind} - \vv{\levtwo}^{\repind}  \right)}
\newcommand{\vecdiffcentone}{\left( \erep{\vv{\model}} - \vv{\levone} \right)}

% diagonal
\newcommand{\diag}[1]{\hat{#1}}
\newcommand{\coveig}{\sigma^{2}}

% quality of life
\newcommand{\ndata}{N_{\rm data}}
\newcommand{\nfits}{N_{\rm fits}}
\newcommand{\nreps}{N_{\rm replicas}}
\newcommand{\modelstd}{\hat{\sigma}}
\newcommand{\erf}{{\rm erf}}
\newcommand{\repind}{(k)}
\newcommand{\cov}{C}
\newcommand{\invcov}[1]{\cov^{-1}_{#1}}
\newcommand{\erep}[1]{\mathbf{E}_{\noise}\left[ #1 \right]}
\newcommand{\eshift}[1]{\mathbf{E}_{\shift}\left[ #1 \right]}
\newcommand{\nfit}{\texttt{n3fit}}
\newcommand{\repchis}{{\chi^2}^{\repind}}
\newcommand{\ie}{{\it i.e.}}
\newcommand{\eg}{{\it e.g.}}
\newcommand{\viz}{{\it viz.}}

\newcommand{\likelihood}{\mathcal{L}}
\newcommand{\logll}{l}

% Difference vectors - might want to change these
\newcommand{\diffreptwo}{\left( \model^{\repind} - \levtwo^{\repind} \right)}
\newcommand{\diffcentone}{\left( \erep{\model} - \levone \right)}
\newcommand{\diffcentunder}{\left( \erep{\model} - \law \right)}
\newcommand{\diffcentrep}{\left( \erep{\model} - \model^{\repind}\right)}
% discard below?
\newcommand{\levelonediff}{\Delta}
\newcommand{\underlyingdiff}{u}
\newcommand{\repdiff}{v}

% estimators
\newcommand{\bias}{{\rm bias}}
\newcommand{\var}{{\rm variance}}
\newcommand{\covrep}{C^{(\rm replica)}}
\newcommand{\covcent}{C^{(\rm central)}}
\newcommand{\shiftcross}{{\rm shift \, cross \, term}}
\newcommand{\noisecross}{{\rm noise \, cross \, term}}
\newcommand{\deltaeps}{\Delta_{\epsilon}}
\newcommand{\kldiv}{D_{KL}}
%polynomial model
\newcommand{\nlaw}{N_{\rm law}}
\newcommand{\nmodel}{N_{\rm model}}
%PDF space
\newcommand{\npoints}{N_{\rm points}}


\graphicspath{{./figures/}}

\title{Bayesian Approach to Inverse Problems: an Application to NNPDF Closure Testing}
\author[a]{Luigi Del Debbio} 
\author[b]{Tommaso Giani} 
\author[a]{Michael Wilson}
\affil[a]{Higgs Centre for Theoretical Physics, School of Physics and Astronomy,
Peter~Guthrie~Tait~Road, Edinburgh EH9 3 FD, United Kingdom.}
\affil[b]{Nikhef Theory Group, Science Park 105, 1098 XG Amsterdam, The Netherlands}


\makeindex

\begin{document}

\maketitle

\begin{abstract}
    We discuss the Bayesian approach to the solution of inverse problems and
    apply the formalism to analyse the closure tests performed by the NNPDF
    collaboration. Starting from a comparison with the approach that is
    currently used for the determination of parton distributions (PDFs) by the
    NNPDF collaboration, we discuss some analytical results that can be obtained
    for linear problems and use these results as a guidance for the more
    complicated non-linear problems. We show that, in the case of Gaussian
    distributions, the posterior probability density of the parametrized PDFs is
    fully determined by the results of the NNPDF fitting procedure. In the
    particular case that we consider, the fitting procedure and the Bayesian
    analysis yield exactly the same result. Building on the insight that we
    obtain from the analytical results, we introduce new estimators to assess
    the statistical faithfulness of the fit results in closure tests. These
    estimators are defined in data space, and can be studied analytically using
    the Bayesian formalism in a linear model in order to clarify their meaning.
    Finally we present numerical results from a number of closure tests
    performed with current NNPDF methodologies. These further tests allow us to
    validate the new methodology and provide a quantitative comparison of the
    new and old methodologies. As PDFs determinations move into precision
    territory, the need for a careful validation of the methodology becomes
    increasingly important: the error bar has become the focal point of
    contemporary PDFs determinations. In this perspective, theoretical
    assumptions and other sources of error are best formulated and analysed in
    the Bayesian framework, which provides an ideal language to address the
    precision and the accuracy of current fits. 

\end{abstract}

\section{Introduction}
\label{sec:Intro}

{\bf to be written. Keeping the text below for reference}

In an NNPDF fit to experimental data, a set of replicas are fitted to pseudodata
which is generated according the experimental central values and uncertainties.
A successful fit is realised if the resulting replicas agree with the underlying
law within uncertainties. Since the underlying law is usually unknown we cannot
directly measure if this is the case.

We can, however, test the methodology in a fit to artificial data, which is
generated from theory predictions from an input PDF and check that we agree with
the input PDF within uncertainties. This is the basis of a closure test, which
has already been used to test to validity of previous iterations of the NNPDF
methodology. Here we aim to refine some of the pre-existing closure test
estimators and with the help of fast fitting methodology perform a more
extensive study of how faithful our uncertainties are.


\section{Inverse Problems}
\label{sec:inverse-problems}

The problem of determining PDFs from a set of experimental data falls under the
general category of {\em inverse problems}, \ie\ the problem of finding the
input to a given model knowing a set of observations, which are often finite and
noisy. In this Section we are going to review the Bayesian formulation of
inverse problems. It is impossible to do justice to vast subject here. Instead
we try to emphasise the aspects that are relevant for quantifying uncertainties
on PDF determinations. 

\subsection{Statement of the problem}
\label{sec:BayesianInverse}

The space of inputs is denoted by $\modelspace$, while $R$ denotes the space of responses.
The model is specified by a {\em forward map}
\begin{align}
  \label{eq:ForwardMap}
  \fwdmapop : ~& \modelspace \to R \nonumber \\
      & \modelvec \mapsto r=\fwdmapop(\modelvec) \, ,
\end{align}
which associates a response $r \in R$ to the input $\modelvec \in \modelspace$, where we assume
that $\modelspace$ and $R$ are Banach spaces.~\footnote{Banach spaces are complete normed
vector spaces. We do not need to get into a more detailed discussion here, but
it is important to note that working in Banach spaces allows us to generalise
the results to infinite-dimensional spaces of functions.} As an example we can
think of $\modelvec$ as being a Parton Distribution Function, \ie\ a function defined on
the interval $[0,1]$, and $r$ a DIS structure function,
\begin{align}
  \label{eq:DISExample}
  r(x,Q^2) = \int_x^1 \frac{dz}{z}\, C(z,Q^2) \modelvec(x/z,Q^2)\, .
\end{align}
Note that in this example the forward map maps one real function into another
real function. Experiments will not have access to the full function $r$ but
only to a subset of $\ndata$ observations. In order to have a formal
mathematical expression that takes into account the fact that we have a finite
number of measurements, we introduce an {\em observation operator}
\begin{align}
  O : ~& R \to Y \nonumber \\
       & r \mapsto \obs \, ,
\end{align}
where $\obs \in Y$ is a vector in a finite-dimensional space $Y$ that contains
all the experimental results, \eg\ the value of the structure function for
different values of the kinematics. In general we will assume that $\obs \in
\real^{\ndata}$, \ie\ we have a finite number $\ndata$ of real experimental
values. The quantity of interest is the composed operator
\begin{align}
  \fwdobsop : ~& \modelspace \to \real^{\ndata} \nonumber \\
                 & \fwdobsop = O \circ G\, ,
\end{align}
which maps the input $\modelvec$ to the set of data. Taking into account the fact that
experimental data are subject to noise, we can write
\begin{align}
  \label{eq:NoisyInverseProblem}
  \obs = \fwdobsop(\modelvec) + \obsnoise\, ,
\end{align}
where $\obsnoise$ is a random variable defined over $\real^{\ndata}$
with probability density $\rho(\obsnoise)$. We will refer to $\obsnoise$ as the
{\em observational noise}. The inverse problem becomes finding $\modelvec$
given $\obs$. It is often the case that inverse problems are ill-defined
in the sense that the solution may not exist, may not be unique, or
may be unstable under small variations of the problem. 

In solving this problem we are going to adopt a Bayesian point of view; our
prior knowledge about $\modelvec$ is encoded in a prior probability measure
$\mu_M^0$, and we use Bayes' theorem to compute the posterior probability
measure of $\modelvec$ given the data $\obs$, which we denote as
$\mu_M^\fwdobsop$. When possible, we denote the probability densities associated
to $\mu_M^0$ and $\mu_M^\fwdobsop$, by $\pi_M^0$ and $\pi_M^\fwdobsop$
respectively. Then, using Eq.~(\ref{eq:NoisyInverseProblem}), we can write the
data likelihood, \ie\ the probability density of $\obs$ given $\modelvec$,
\begin{align}
  \label{eq:YGivenUProbDensity}
  \pi(\obs|\modelvec) = \rho(\obs-\fwdobsop(\modelvec))\, ,
\end{align}
and Bayes' theorem yields
\begin{align}
  \label{eq:BayesThmInversePosterior}
  \pi_M^\fwdobsop(\modelvec) = \pi(\modelvec|\obs) \propto \rho(\obs-\fwdobsop(\modelvec)) \pi_M^0(\modelvec)\, .
\end{align}

\paragraph{Example}

Even though the concepts that we have introduced so far should sound familiar,
it is worthwhile to spend a few paragraphs to clarify some ideas and present an
explicit argument, where all the probability densities are carefully defined.
This is best exemplified by considering the case where both the observational
noise and the model prior are Gaussian. We assume that we are given a set of
central values $\obspriorcent \in \real^{\ndata}$ and their covariance matrix
$\obspriorcov$. Then the {\em prior} probability density of the observable
$\obs$ is 
\begin{equation}
  \label{eq:PriorData}
  \pi_{D}^0(\obs|\obspriorcent,\obspriorcov) \propto \exp\left(
    -\frac12 \left| \obs - \obspriorcent \right|_{\obspriorcov}^2
    \right)\, ,
\end{equation}
where the suffix $D$ emphasises the fact that this is a probability density in
data space, and the notation explicitly reminds us that this is the probability
density given the central values $\obspriorcent$ (and the covariance matrix).
Similarly we can choose a Gaussian distribution for the input model,
characterized by a central value $\modelpriorcent$ and a covariance
$\modelpriorcov$:
\begin{align}
  \label{eq:PiZeroGauss}
  \pi_{M}^0(\modelvec|\modelpriorcent,\modelpriorcov)  
  &\propto \exp\left(
              -\frac12 \left| \modelvec - \modelpriorcent \right|_{\modelpriorcov}^2
              \right)\, .
\end{align}
Following the convention above, we use a suffix $M$ here to remind the reader
that we are looking at a probability density in the space of models. Note that
in the expressions above we used the norms in $\modelspace$ and $\real^{\ndata}$
respectively, and introduced the short-hand notation
\begin{align}
  \left|a\right|_M^2 = \left| M^{-1/2} a\right|^2\, ,
\end{align}
where $a$ denotes a generic element of $\modelspace$, $R$ or $\real^{\ndata}$.
For the case where $a \in \real^{\ndata}$, we use the Euclidean norm and
\begin{align}
  \left| a \right|_M^2 = \sum_{i,j} a_i M_{ij} a_j\, ,
\end{align}
where the indices $i,j$ run from 1 to $\ndata$.  
Up to this point data and models are completely independent, and the joint
distribution is simply the product of $\pi_{D}^0$ and $\pi_{M}^0$. 

The forward map induces a correlation between the input model and the
observables, so we introduce a probability density $\theta$ that describes these
correlations due to the physical theory,  
\begin{equation}
  \label{eq:ThetaCorr}
  \theta(\obs,\modelvec|\fwdobsop) = \delta\left(\obs - \fwdobsop(\modelvec)\right)\, ,
\end{equation}
where the Dirac delta corresponds to the case where there are no theoretical
uncertainities. Theoretical uncertainties can be introduced by broadening the
distribution of $\obs$ away from the exact prediction of the forward map, \eg\
using a Gaussian with covariance $C_T$,
\begin{equation}
  \label{eq:TheoryErrors}
  \theta(\obs,\modelvec|\fwdobsop) = \exp\left(
    -\frac12 
    \left| \obs - \fwdobsop(\modelvec)
    \right|_{C_T}^2\right)\, .
\end{equation}
Note however that there are no rigorous arguments favouring the fact that
theoretical errors are normally distributed. Taking the correlation
$\theta(\obs,\modelvec|\fwdobsop)$ into account, the joint distribution of
$\obs$ and $\modelvec$ is
\begin{align}
  \label{eq:JointYAndU}
  \pi^\fwdobsop(\obs,\modelvec|\obspriorcent,\obspriorcov,\modelpriorcent,\modelpriorcov) 
  \propto 
  \pi_{D}^0(\obs|\obspriorcent,\obspriorcov) 
  \pi_{M}^0(\modelvec|\modelpriorcent, \modelpriorcov) 
  \theta(\obs,\modelvec|\fwdobsop)\, .
\end{align}
We can now marginalize with respect to \obs, neglecting theory errors, 
\begin{align}
  \label{eq:MarginOne}
  \pi^\fwdobsop_M(\modelvec|\obspriorcent,\obspriorcov,\modelpriorcent,\modelpriorcov) 
  &\propto \int dy\, \pi_{D}^0(\obs|\obspriorcent,\obspriorcov) 
    \pi_{M}^0(\modelvec|\modelpriorcent,\modelpriorcov) 
    \theta(\obs,\modelvec|\fwdobsop) \\
  & \propto \pi_{M}^0(\modelvec|\modelpriorcent,\modelpriorcov)  
    \int dy\, \pi_{D}^0(\obs|\obspriorcent,\obspriorcov) 
    \delta\left(\obs-\fwdobsop(\modelvec)\right) \\
  & \propto \pi_{M}^0(\modelvec|\modelpriorcent,\modelpriorcov) 
    \pi_{D}^0(\fwdobsop(\modelvec)|\obspriorcent,\obspriorcov)\, .
\end{align}
The log-likelihood in the Gaussian case is simply the $\chi^2$ of the data,
$\obspriorcent$, to the theory prediction, $\fwdobsop(\modelvec)$:
\begin{equation}
  \label{eq:LikelyChiSq}
  -\log\pi_D^0(\fwdobsop(\modelvec)|\obspriorcent) =  
      \frac12 \left|
      \fwdobsop(\modelvec) - \obspriorcent
      \right|_{\obspriorcov}^2
    \, .
\end{equation}
In the notation of Eq.~\ref{eq:BayesThmInversePosterior}
\begin{equation}
  \label{eq:IdentifyRho}
  \pi_D^0(\fwdobsop(\modelvec)|\obspriorcent) = \rho\left(
    \fwdobsop(\modelvec) - \obspriorcent
  \right)\, ,
\end{equation}
where in this case 
\begin{align}
  \label{eq:RhoGauss}
  \rho(\obsnoise) &\propto \exp\left(
               -\frac12 \left|\obsnoise\right|_{\obspriorcov}^2
               \right)\, .
\end{align}
The probability density
$\pi^\fwdobsop_M(\modelvec|\obspriorcent,\obspriorcov,\modelpriorcent,\modelpriorcov)$
was called $\pi_M^\fwdobsop(\modelvec)$ in
Eq.~\ref{eq:BayesThmInversePosterior}, where the suffix $\fwdobsop$ is a
short-hand to denote the posterior probability in model space, taking into
account all the conditional variables. Hence, for the Gaussian case, the result
from Bayes' theorem reduces to
\begin{align}
  \label{eq:PosteriorModel}
  \pi_M^\fwdobsop(\modelvec) 
  &\propto 
  \exp\left[
    -\frac12 \left| \obspriorcent - \fwdobsop(\modelvec) \right|_{\obspriorcov} ^2
    -\frac12 \left| \modelvec - \modelpriorcent \right|_{\modelpriorcov}^2
  \right] \\ 
  &= 
  \exp\left[
    - S(\modelvec)
  \right]\, .
\end{align}
Note that in the argument of the likelihood function we have the central values
of the data points $\obspriorcent$. Eq.~(\ref{eq:PosteriorModel}) is the
Bayesian answer to the inverse problem, our knowledge of the model $\modelvec$
is encoded in the probability measure $\mu_M^\fwdobsop$, which is fully
specified by the density $\pi_M^\fwdobsop$. There are several ways to
characterise a probability distribution. The NNPDF approach is focused on the
determination of the {\em Maximum A Posteriori (MAP)} estimator, \ie\ the
element $u_* \in \modelspace$ that maximises $\pi_M^\fwdobsop(\modelvec)$:
\begin{align}
  \label{eq:MAP}
  u_* = \arg\min_{\modelvec \in \modelspace} 
  \left(
  -\frac12 \left| \obspriorcent - \fwdobsop(\modelvec) \right|_{\obspriorcov}^2
  -\frac12 \left| \modelvec - \modelpriorcent \right|_{\modelpriorcov}^2
  \right)\, .
\end{align}
For every instance of the data $\obspriorcent$, the MAP estimator is computed by
minimising a regulated $\chi^2$, we will refer to this procedure as the {\em
classical fit} of experimental data to a model. Note that in the Bayesian
approach, the regulator appears naturally after having specified carefully all
the assumptions that enter in the prior. In this specific example the regulator
arises from the Gaussian prior for the model input $\modelvec$, which is
normally ditributed around a solution $\modelpriorcent$. The MAP estimator
provides the explicit connection between the Bayesian approach and the classical
fit.

\subsection{Comparison with classical fitting}
\label{sec:comp-class-fit}

There are a number of results that make the connection between the two
approaches more quantitative, and therefore more transparent. We are going to
summarise these results here without proofs, referring the reader to the
mathematical literature for the missing details. Working in the
finite-dimensional case, we assume 
\begin{align*}
  \modelvec &\in \real^{\nmodel} \, ,\\
  \obs &\in \real^{\ndata}\, ,
\end{align*}
and we are going to consider in detail two examples from Ref.~\cite{StuartCore},
which illustrate the role of the priors in the Bayesian setting. It is
particularly useful to distinguish the case of an underdetermined system from
the case of an overdetermined one. 

\paragraph{Underdetermined system}
The first case that we are going to analyse is the case of a linear system that
is underdetermined by the data. The linear model is completely specified by a
vector of coefficients $g\in \real^{\nmodel}$, 
\begin{equation}
  \label{eq:LinSyst}
  \mathcal{G}(u) = \left(g^T u\right)\, .
\end{equation}
Assuming that we have only one datapoint, \ie\ $\ndata=1$, 
\begin{equation}
  \label{eq:LinearModelEx}
  \obs = (g^T \modelvec) + \obsnoise\, ,
\end{equation}
where $\obsnoise \sim \mathcal{N}(0,\gamma^2)$ is one Gaussian number, whose
probability density is centred at $0$ and has variance $\gamma^2$. For
simplicity we are going to assume that the prior on $\modelvec$ is also a
multi-dimensional Gaussian, centred at $0$ with covariance matrix $\modelpriorcov$. In
this case the posterior distribution can be written as
\begin{equation}
  \label{eq:GaussPostExplicit}
    \pi_M^\fwdobsop(\modelvec) 
    \propto \exp \left[
      -\frac{1}{2\gamma^2} \left|\obs - (g^T \modelvec) \right|^2 - \frac12 \left|\modelvec\right|_{\modelpriorcov}^2 
    \right]\, ,
\end{equation}
which is still a Gaussian distribution for $\modelvec$. The mean and covariance
are respectively
\begin{align}
  m &= \frac{(\modelpriorcov g) \obs}{\gamma^2 + (g^T \modelpriorcov g)}\, , \\
  \Sigma &= \modelpriorcov - 
  \frac{(\modelpriorcov g) (\modelpriorcov g)^T}{\gamma^2 + (g^T \modelpriorcov g)}\, .
\end{align}
Because the argument of the exponential is a quadratic form, the mean of the
distribution coincides with the MAP estimator. It is instructive to look at
these quantities in the limit of infinitely precise data, \ie\ in the limit
$\gamma\to 0$:
\begin{align}
  m_\star &= 
  \lim_{\gamma\to 0} m
  = \frac{(\modelpriorcov g) \obs}{(g^T \modelpriorcov g)}\, , \\
  \Sigma_\star &= 
  \lim_{\gamma\to 0} \Sigma 
  = \modelpriorcov - 
  \frac{(\modelpriorcov g) (\modelpriorcov g)^T}{(g^T \modelpriorcov g)}\, .
\end{align}
These values satisfy
\begin{align}
  (g^T m_\star) = \obs \, , \\
  (\Sigma_\star g) = 0 \, ,
\end{align}
which shows that the mean of the distribution is such that the data point is
exactly reproduced by the model, and that the uncertainty in the direction
defined by $g$ vanishes. It should be noted that the uncertainty in directions
perpendicular to $g$ does not vanish and is determined by a combination of the
prior and the model, \viz\ $\modelpriorcov$ and $g$ in our example. This is a
particular example of a more general feature: for underdetermined systems the
information from the prior still shapes the probability distribution of the
solution even in the small noise limit.  

\paragraph{Overdetermined system}
We are now going to consider an example of an overdetermined system and discuss
again the case of small observational noise. We consider $\ndata\geq 2$ and
$n=1$, with a linear forward map such that
\begin{equation}
 \label{eq:OverDetForwMap}
 \obs = g \modelvec  + \obsnoise\, ,
\end{equation} 
where $\obsnoise$ is an $\ndata$-dimensional Gaussian variable with a diagonal
covariance $\gamma^2 I$, where $I$ denotes the identity matrix. For simplicity
we are going to assume a Gaussian prior with unit variance for $\modelvec$, which yields
for the posterior distribution:
\begin{equation}
  \label{eq:OverDetPost}
  \pi_M^\fwdobsop(\modelvec) 
    \propto 
    \exp\left(
      -\frac{1}{2\gamma^2} \left| \obs - g(\modelvec + \beta \modelvec^3)\right|^2
      -\frac12 \modelvec^2
    \right)\, .
\end{equation} 
If $\beta=0$ the posterior is Gaussian and we can easily compute its mean and variance: 
\begin{align}
  m &= \frac{(g^T \obs)}{\gamma^2 + |g|^2} \, , \\
  \sigma^2 &=
    \frac{\gamma^2}{\gamma^2 + |g|^2}\, .
\end{align}
In this case, in the limit of vanishing observational noise, we obtain
\begin{align}
  m_\star &= \frac{(g^T \obs)}{|g|^2} \, ,\\
  \sigma_\star^2 &= 0\, .
\end{align}
The mean is given by the weighted average of the datapoints, which is also the solution of the $\chi^2$ minimization
\begin{equation}
  m_\star = \arg\min_{\modelvec\in\real} \left|\obs - g \modelvec\right|^2\, .
\end{equation}
Note that in this case the variance $\sigma_\star$ vanishes independently of the
prior. In the limit of small noise, the distribution tends to a Dirac delta
around the value of the MAP estimator.  

\subsection{Linear Problems}
\label{sec:LinProbs}

Linear problems in finite-dimensional spaces are characterized by a simple forward law, 
\begin{equation}
  \label{eq:MatrixG}
  \obs = \linmap \modelvec\, ,
\end{equation}
where $\linmap$ is a matrix. In this framework one can readily  derive
analytical solutions that are useful to understand the main features of the
Bayesian approach. Assuming that the priors are Gaussian again, the cost
function $S(\modelvec)$ is a quadratic function of $\modelvec$,
\begin{align}
  \label{eq:CostLinGauss}
  S(\modelvec) &= 
  \left(\linmap \modelvec - \obspriorcent \right)^T \obspriorcov^{-1} 
  \left(\linmap \modelvec - \obspriorcent \right) + 
  \left( \modelvec - \modelpriorcent \right)^T \modelpriorcov^{-1} \left(\modelvec - \modelpriorcent \right) \\
  &= 
  \left(\modelvec - \modelpostcent\right) \modelpostcov^{-1}
  \left(\modelvec - \modelpostcent\right) + \mathrm{const}\, ,
\end{align} 
where
\begin{align}
  \label{eq:PostParamsCov}
  \modelpostcov^{-1} &= 
  \left(
    \linmap^T \obspriorcov^{-1} \linmap + \modelpriorcov^{-1}
  \right)\, , \\
  \label{eq:PostParamsMean}
  \modelpostcent &=
  \modelpostcov  \left(
    \linmap^T \obspriorcov^{-1} \obspriorcent + \modelpriorcov^{-1} \modelpriorcent
  \right)\, .
\end{align}
The case where we have no prior information on the model is recovered by taking
the limit $\modelpriorcov^{-1} \to 0$, which yields
\begin{align}
  \label{eq:NoPriorLinModel}
  \modelpostcov^{-1} &= 
  \left(
    \linmap^T \obspriorcov^{-1} \linmap
  \right)\, , \\
  \modelpostcent &=
  \modelpostcov  \left(
    \linmap^T \obspriorcov^{-1} \obspriorcent 
  \right)\, . \label{eq:NoPriorLinModelCov}
\end{align}
The action of $\obspriorcov^{-1}$ on the vector of observed data $\obspriorcent$
is best visualised using a spectral decomposition
\begin{equation}
  \label{eq:CDSpecDec}
  \obspriorcov^{-1} = \sum_n \frac{1}{\sigma_n^2} P_n\, ,
\end{equation}
where $P_n$ denotes the projector on the $n$-th eigenvector of $\obspriorcov$, and
$\sigma_n^2$ is the corresponding eigenvalue. The action of $\obspriorcov^{-1}$ is to
perform a weighted average of the components of $\obspriorcent$ in the directions of the
eigenvectors of $\obspriorcov$.

An explicit expression for the posterior distribution of data can be obtained
from the joint distribution by marginalising over the model input $\modelvec$:
\begin{align}
  \label{eq:PosteriorDataSpace}
  \pi_D^\fwdobsop(\obs|\obspriorcent,\obspriorcov,\modelpriorcent,\modelpriorcov)
  &= \int du\, 
  \pi^\fwdobsop(\obs,\modelvec|\obspriorcent,\obspriorcov,\modelpriorcent,\modelpriorcov) \\
  &\propto \exp\left(
    -\frac12 \left(\obs - \obspostcent\right)^T \obspostcov^{-1}
    \left(\obs - \obspostcent\right)
  \right)\, ,
\end{align}
where
\begin{align}
  \label{eq:PosteriorDataParamsMean}
  \obspostcent &= \linmap \modelpostcent\, , \\
  \label{eq:PosteriorDataParamsCov}
  \obspostcov &= \linmap \modelpostcov \linmap^T\, .
\end{align}

\paragraph{Posterior distribution of unseen data}

In real-life cases we are also interested in the posterior distribution of a set
of data that have not been included in the fit. In the Bayesian framework that
we have developed this can be modeled by having two sets of data $y$ and $y'$,
for which we have a prior distribution 
\begin{align}
  \label{eq:JointIndepDataPrior}
  \pi_D^0&\left(y,y'|y_0,C_D,y'_0,C'_D\right) 
   = \pi_D^0\left(y'|y'_0,C'_D\right) \pi_D^0\left(y|y_0,C_D\right) \\
  & \propto 
  \exp\left[-\frac12 \left(y'-y'_0\right)^T (C'_D)^{-1} 
  \left(y'-y'_0\right)\right]\, 
  \exp\left[-\frac12 \left(y-y_0\right)^T (C_D)^{-1} 
  \left(y-y_0\right)\right]\, .
\end{align}
Following the derivation above, we can write the joint distribution for the data
and the model 
\begin{equation}
  \label{eq:JointModelData}
  \pi^\fwdobsop(y,y',u) 
  \propto 
  \pi_D^0(y,y'|y_0,C_D,y'_0,C'_D) \pi_M^0(u) 
  \delta\left(y - \mathcal{G}u\right)
  \delta\left(y'- \mathcal{G}'u\right)\, .
\end{equation}
Note that because both sets of data are derived from the same model $u$, the
joint distribution above introduces a correlation between the data sets. 

We can now marginalise with respect to the dataset $y$, 
\begin{equation}
  \label{eq:MarginaliseDatasetY}
  \begin{split}
    \pi(y',u) 
    \propto &
    \exp\left[-\frac12 \left(y'-y'_0\right)^T (C'_D)^{-1} 
    \left(y'-y'_0\right)\right]\, 
    \exp\left[-\frac12 \left(u-\tilde{u}\right)^T (\tilde{C}_M)^{-1} 
    \left(u-\tilde{u}\right)\right] \\
    & \quad \times \delta\left(y'- \mathcal{G}'u\right)\, .
  \end{split}
\end{equation}
where $\tilde{C}_M$ and $\tilde{u}$ are given respectively in
Eqs.~\ref{eq:PostParamsCov} and \ref{eq:PostParamsMean}. By marginalising again,
this time with respect to the model, we derive the posterior distribution of the
unseen data,
\begin{equation}
  \label{eq:MarginaliseModelU}
  \pi^y_D(y') \propto 
  \exp \left[ 
   \frac12 \left(y' - \tilde{y}'\right)^T
   (\tilde{C}'_D)^{-1} 
   \left(y' - \tilde{y}'\right)
   \right]\, ,
\end{equation}
where
\begin{align}
  \tilde{C}'_D 
  &= \mathcal{G}' \tilde{C}'_M \mathcal{G}'^{T} \\
  \tilde{y}'
  &= \mathcal{G}' \tilde{u}'\, ,
\end{align}
and 
\begin{align}
  \tilde{C}_M'^{-1} 
  &= \mathcal{G}'^T C_D'^{-1} \mathcal{G}' + \tilde{C}_M^{-1} \\
  \tilde{u}' 
  &= \tilde{C}'_M \left(
    \mathcal{G}'^T C_D'^{-1} y_0' + \tilde{C}_M^{-1} \tilde{u} 
    \right) \\
  \tilde{C}_M^{-1}
  &= \mathcal{G}^T C_D^{-1} \mathcal{G} + C_M^{-1} \\
  \tilde{u}
  &= \tilde{C}_M \left(
    \mathcal{G}^T C_D^{-1} y_0 + C_M^{-1} u_0
  \right)\, .
\end{align}

\paragraph{Non-linear Models}

The linear models that we have discussed so far may look over-simplified at
first sight. In practice, it turns out that non-linear models can often be
linearised around the central value of the prior distribution, 
\begin{equation}
  \label{eq:LinU0}
  \fwdobsop(\modelvec) = \fwdobsop(\modelpriorcent) + G \left(\modelvec - \modelpriorcent\right) + \ldots\, ,
\end{equation}
where 
\begin{equation}
  \label{eq:FirstDerU0}
  G^i_\alpha = \left. \frac{\partial \fwdobsop^i}{\partial u_\alpha} \right|_{\modelpriorcent}\, ,
\end{equation}
and we have neglected higher-order terms in the expansion of
$\fwdobsop(\modelvec)$.

If these terms are not negligible, another option is to find the MAP estimator,
and then expand the the forward map around it, which yields equations very
similar to the previous ones, with $\modelpriorcent$ replaced by $u_*$. If the
posterior distribution of $u$ is sufficiently peaked around the
MAP estimator, then the linear approximation can be sufficiently accurate. 


\subsection{The infinite-dimensional case}
\label{sec:infin-dimens-case}

In the finite-dimensional case, where the probability measures are specified by
their densities with respect to the Lebesgue measure,
Eq.~(\ref{eq:BayesThmInversePosterior}) can be rephrased by saying  that $\rho$
is the Radon-Nikodym derivative of the probability measure $\mu^\fwdobsop$ with
respect to $\mu_0$, \viz
\begin{align}
  \label{eq:RadonNikodym}
  \frac{d\mu^\fwdobsop}{d\mu^0} (\modelvec) \propto \rho(\obs-\fwdobsop(\modelvec)) \pi^0(\modelvec)\, .
\end{align}
Finally, using the fact that the density $\rho$ is a positive
function, we can rewrite 
\begin{align}
  \label{eq:PotentialDef}
  \rho(\obs-\fwdobsop(\modelvec)) = \exp\left(-\Phi(\modelvec;\obs)\right)\, ,
\end{align}
and therefore
\begin{align}
  \label{eq:RadonNikodymTwo}
  \frac{d\mu^\fwdobsop}{d\mu^0} (\modelvec) \propto \exp\left(-\Phi(\modelvec;\obs)\right)\, .
\end{align}
In finite-dimensional spaces, the three equations above are just definitions
that do not add much content. Their interest resides in the fact that the last
expression, Eq.~(\ref{eq:RadonNikodymTwo}), can be properly defined when $\modelspace$ is
infinite-dimensional, allowing a rigorous extension of the Bayesian formulation
of inverse problems to the case of infinite-dimensional spaces. 

Adopting a heuristic approach, we can say that a function $f$ is a random
function is $f(x)$ is a random variable for all values of $x$. Since the values
of the function at different values of $x$ can be correlated, a random function
is fully characterised by specifying the joint probability densities
\begin{equation}
  \label{eq:RandomFuncJointProb}
  \pi\left(
    f_1, \ldots, f_n; x_1, \ldots x_n
  \right)\, ,
\end{equation}
where $f_i=f(x_i)$, for all values of $n$, and all values of $x_1, \ldots, x_n$.
These finite-dimensional densities allow the definition of a probability
measure, under some regularity hypotheses that we will not investigate here.
Similarly to what happens in the finite-dimensional case, a Gaussian random
function is completely characterised by the two-dimensional probability
densities $\pi(x_1,x_2;t_1,t_2)$.

\paragraph{Functional linear problems} This formalism allows us to formulate a Bayesian solution of the inverse problem 
\begin{equation}
  \label{eq:BayesLinearInverse}
  y^i = \int dx\, G^i(x) u(x)\, ,
\end{equation}
where $y_i$ is a discrete set of oobservables and $u(x)$ is Gaussian random
function, with prior mean $u_0(x)$ and covariance $C_M(x,x')$. The vector of
observed values is denoted $y_0$, and we assume that the prior distribution of
$y$ is a Gaussian centred at $y_0$ with covariance $C_D$. We introduce the
matrix
\begin{equation}
  \label{eq:Smatrix}
  S^{ij} =
  \int dx dx'\, G^i(x) C_M(x,x') G^j(x') + C_D^{ij}\, ,
\end{equation}
and its inverse $T_{ij}=\left(S^{-1}\right)^{ij}$. Just like in the finite-dimensional example discussed above, the posterior Gaussian field is centred at
\begin{align}
  \label{eq:PostMeanFunc}
  \tilde{u}(x) = u_0(x) + 
  \int dx'\, C_M(x,x') G^i(x') T_{ij} \left(
    y_0^j - \int dx''\, G^j(x'') u_0(x'') 
  \right)\, .
\end{align}
Interestingly, this can be rewritten as
\begin{equation}
  \label{eq:TowardsBackus}
  \tilde{u}(x) = u_0(x) + 
  \int dx'\, C_M(x,x') \Psi(x')\, ,
\end{equation}
where 
\begin{eqnarray}
  \label{eq:PsiDef}
  \Psi(x) = G^i(x) \delta y^i\, ,
\end{eqnarray}
and the weighted residuals are given by
\begin{equation}
  \label{eq:DeltaYDef}
  \delta y^i = T_{ij} \left(
  y_0^j - \int dx'\, G^j(x') u_0(x')
  \right)\, .
\end{equation}
Using the notation introduced above, the posterior covariance can be written as
\begin{equation}
  \label{eq:PostCovFunc}
  \tilde{C}_M(x,x') = 
  C_M(x,x') - \Psi^i(x) T_{ij} \Psi^j(x')\, .
\end{equation}

The Bayesian result summarised above can be compared with the method proposed by Backus and Gilbert to solve the same problem. Assuming that there exists an unknown 'true' model $\utrue$, such that the observed data are
\begin{equation}
  \label{eq:BackStart}
  y_0^i = \int dx\, G^i(x) \utrue\, ,
\end{equation}
we look for an estimate $\uest$ of the true solution in the form
\begin{equation}
  \label{eq:BackAnsatz}
  \uest(x) = Q^i(x) y^i_0\, .
\end{equation}
Using Eq.~\ref{eq:BackStart} we obtain
\begin{equation}
  \label{eq:BackFilter}
  \uest(x) = \int dx' R(x,x') \utrue(x')\, , 
\end{equation}
which states that with a finite amount of data we can only resolve a filtered
version of the true solution. The kernel $R$ is given by
\begin{equation}
  \label{eq:BackKernel}
  R(x,x') = Q^i(x) G^i(x')\, .
\end{equation}
The coefficients $Q(x)$ can be chosen so that the kernel is as close as possible
to a delta function:
\begin{equation}
  \label{eq:BackDelta}
  R(x,x') \simeq \delta(x,x') ~~ \Longrightarrow ~~
  \uest \simeq \utrue\, .
\end{equation}
Approximating the delta function can be achived by minimising 
\begin{equation}
  \label{eq:BackDeltaness}
  \int dx'\, \left(
    R(x,x') - \delta(x-x')
  \right)^2\, ,
\end{equation}
which yields
\begin{equation}
  \label{eq:BackSolution}
  Q^i(x) = \left(S^{-1}\right)^{ij} G^j(x)\, ,
\end{equation}
where 
\begin{equation}
  \label{eq:BackSMatrix}
  S^{ij} = \int dx\, G^i(x) G^j(x)\, .
\end{equation}
The intersting observation is that the central value of the Bayesian solution
presented above reduces to the Backus-Gilbert $\uest$ in the case where there is
no prior for the model, \ie\ if we assume that $u_0$ is just white noise and
therefore
\begin{equation}
  \label{eq:BackComparison}
  C_M(x,x') = \delta(x-x')\, .
\end{equation}
The Bayesian solution allows a variety of prior to be explicitly declared and
tested compared to the Backus-Gilbert solution. 
\section{NNPDF Monte Carlo approach to inverse problems}
\label{sec:closure-test}

In this section we will discuss the NNPDF approach to inverse problems, trying
to make contact explicitly with the formalism laid out in
Sec.~\ref{sec:inverse-problems}. In particular, Eq.~\eqref{eq:PosteriorModel}
gives a formal description of propagating our prior understanding of the data
into model space. In practice, sampling from the posterior distribution is
highly non-trivial.

\subsection{Fitting replicas}
\label{sec:fit-reps}

The approach for generating a sample in model space employed by NNPDF can
broadly be described as fitting model replicas to pseudo-data replicas. As
discussed in Eq.~\eqref{eq:NoisyInverseProblem} the experimental values are
subject to observational noise. If we assume this observational noise to be
multigaussian then the experimental central values, $\obspriorcent$, are given
explicitly by
\begin{equation}
    \label{eq:levelonedata}
    \obspriorcent = \law + \obsnoise,
\end{equation}
where $\law$ is the vector of {\em true} observable values and the obervational
noise is drawn from a Gaussian centered on zero such as in
Eq.~\ref{eq:RhoGauss}, \ie\ $\obsnoise \sim \mathcal{N}(0, \obspriorcov)$ where
$\obspriorcov$ is the experimental covariance matrix. In
Eq.~\eqref{eq:levelonedata}, each basis vector corresponds to a separate data
point, and the vector of shifts $\obsnoise$ permits correlations between data
points according to the covariance matrix provided by the experiments. Given the
data, the NNPDF approach is to compute a MAP estimator similar to that discussed
in the previous section, \ie\ finding the model that minimises the $\chi^2$ to
the data. The key difference between the NNPDF approach and the classical MAP
estimator is instead of fitting the observational data given by
Eq.~\ref{eq:levelonedata}, an ensemble of model replicas are fitted each to an
independently sampled instance of pseudo-data, which is generated by augmenting
$\obspriorcent$ with some noise, $\noise^{\repind}$,
\begin{equation}
    \label{eq:leveltwodata1}
    \pseudodat^{\repind} = \obspriorcent + \noise^{\repind}
    = \law + \obsnoise + \noise^{\repind},
\end{equation}
where $k$ is the replica index and each instance of the noise, $\noise$, is drawn
independently from the same Gaussian from which the observational noise is
drawn from, \ie\ $\noise \sim \mathcal{N}(0, \obspriorcov)$.

The parameters for each model replica maximise the likelihood evaluated on the
corresponding pseudo-data. This is a special case of MAP estimation, described
in Eq.~\eqref{eq:MAP}, where the model prior is uniform. Another way of viewing
this is to take $\modelpriorcov^{-1} \to 0$ in Eq.~\eqref{eq:MAP}, as was done
to obtain the result in Eq.~\ref{eq:NoPriorLinModel}. Either way,
there is no prior information about the model. In this framework, the
parameterisation of the model is fixed, so the model space is the finite space
of parameters $\modelvec \in \real^{\nmodel}$. Furthermore, we actually find the
parameters which minimise the $\chi^2$ between the predictions from the model
and the corresponding pseudo-data $\pseudodat^{\repind}$
\begin{equation}\label{eq:NNPDFLikelihood}
    \begin{split}
        \modelvecrep &= \arg\min_{\modelvec^{\repind}} \repchis \\
        &= \arg\min_{\modelvec^{\repind}} \sum_{ij}
        \left( \fwdobsop(\modelvec^{\repind}) - \pseudodat^{\repind} \right)^T
        \obspriorcov^{-1}
        \left( \fwdobsop(\modelvec^{\repind}) - \pseudodat^{\repind} \right) \, ,
    \end{split}
\end{equation}
where, as usual, minimising the $\chi^2$ is equivalent to maximising the
likelihood, $\likelihood$, since $\chi^2 \equiv -\log{\likelihood}$.
% The uncertainty on the MAP estimator is computed by bootstrapping the
% data, \ie\ by generating an ensemble of pseudo-data, called replicas, that
% fluctuate around $\obspriorcent$. By fitting each replica, we obtain an ensemble
% of models whose distribution is representative of the fluctuations of the MAP
% estimator. The fitted pseudo-data is generated by augmenting the data with some
% noise, $\noise^{\repind}$,

As a final note: since we do not include the model prior, overall normalisations
can be omitted in Eq.~\ref{eq:NNPDFLikelihood}. It is clear however that if we
were including a model prior in our MAP, it is important that the relative
normalisations between the likelihood function and the model prior are
self-consistent.

\subsection{Fluctuations of fitted values}
\label{sec:fluct-fit-values}

It is not immediately obvious that our MC methodology, maximising the likelihood
on an ensemble of pseudo-data replicas, should guarantee that the model replicas
are indeed sampled from the posterior distribution of parameters given data as
described \eg\ in Eq.~\ref{eq:PosteriorModel}. In order to investigate this
issue, we will again consider a model, whose predictions are linear in the model
parameters, where the posterior distribution of model parameters can be written
explicitly. A practical example, which can
elucidate the following arguments would be a polynomial model. Then $\modelvec$ is a
vector of $\nmodel$ polynomial coefficients and $\linmap$ is the Vandermonde matrix
\begin{equation}
    \linmap =
    \begin{bmatrix}
        1  & x_1 & \ldots& x_1^{\nmodel-1} \\ 
        1  & x_2 & \ldots& x_2^{\nmodel-1} \\ 
        \vdots  & \vdots & \vdots& \vdots \\ 
        1  & x_{\ndata} & \ldots & x_{\ndata}^{\nmodel-1} 
    \end{bmatrix}.
\end{equation}
In this case the forward map yields
\begin{equation}
    \label{eq:PolyMod}
    y_i = \sum_{j=0}^{\nmodel-1} u_j x_i^j\, , 
\end{equation}
where $i=1,\ldots,\ndata$. The arguments here are not restricted to polynomials,
however, and apply to any model whose forward map can be expressed as
Eq.~\eqref{eq:MatrixG}, for example a linear shallow approximation of neural
networks \cite{ADVANI2020428}. In order to get an exact analytical solution for
the linear model, we additionally require $\linmap$ to have linearly independent
rows, and therefore $\linmap \obspriorcov \linmap^T$ is invertible. With no
prior information on the model, the posterior distribution of model parameters
is a Gaussian with mean and covariance given by Eqs.~\ref{eq:NoPriorLinModel}
and \ref{eq:NoPriorLinModelCov}.

If instead we deploy the NNPDF Monte Carlo method to fitting model replicas,
then in the case under study $\arg\min_{\modelvec^{\repind}} \repchis$ is found
analytically when the derivative of $\repchis$ with respect to the model
parameters is zero, i.e.
\begin{equation}
    \begin{split}
        \label{eq:MAPEstLinModel}
        \modelvecrep &= (\linmap^T \obspriorcov^{-1} \linmap)^{-1}
        \left(
            \linmap^T \obspriorcov^{-1} \obspriorcent +
            \linmap^T \obspriorcov^{-1} \noise^{\repind}
        \right) \, .
    \end{split}
\end{equation}
Eq.~\ref{eq:MAPEstLinModel} shows that $\modelvec_*$ is a linear
combination of the Gaussian variables $\noise$, and therefore is
also a Gaussian variable. Its
probability density is then completely specified by the average and
covariance of $\modelvec_*$, which can be calculated explicitly, given that the
probability density for $\noise$ is known:
\begin{align}
    \emodel{\modelvec_*} &=
    \modelpostcent = (\linmap^T \obspriorcov^{-1} \linmap)^{-1} \linmap^T
    \obspriorcov^{-1} \obspriorcent \\
    {\rm cov}(\modelvec_*) &= \modelpostcov = (\linmap^T \obspriorcov^{-1} \linmap)^{-1} \, .
\end{align}
In this way, we can show explicitly that under the assumptions specified above,
$\modelvec_* \sim \mathcal{N}(\modelpostcent, \, \modelpostcov)$.
In other words, when the model predictions are linear in the model parameters,
the NNPDF MC method is shown to produce a sample of models from the posterior
distribution of model parameters given the data.

\subsection{Closure test}
\label{sec:closure-test-intro}

The concept of the closure test, which was first introduced in
Ref.~\cite{nnpdf30}, is to construct artificial data by using a known
pre-existing function to generate the {\em true} observable values, $\law$. One
way of achieving this is by choosing $\lawmodel$ such that $\law =
\fwdobsop(\lawmodel)$. Then the experimental central values are artificially
generated according to Eq.~\ref{eq:levelonedata}, except the observational noise
is pseudo-randomly generated from the assumed distribution.

In \cite{nnpdf30}, $\law$ is referred to as level 0 (L0) data and
$\obspriorcent$ is referred to as level 1 (L1) data. Finally, if we use the
NNPDF MC method to fit artificially generated closure data, the pseudo-data
replicas that are fitted by the model replicas are referred to as level 2 (L2)
data.

In a closure test, the assumed prior of the data is fully consistent with the
particular instance of observed central values, $\obspriorcent$, by construction.
In the original closure test in NNDPF3.0 there was also no
modelisation uncertainty, the true observable values were assumed to be obtained
by applying the forward map $\fwdobsop$ to a vector in model space $\lawmodel$.
It is worth noting that the assumption of zero modelisation uncertainties is
quite strong and likely unjustified in many areas of physics. In the context of
fitting parton distribution functions there are potentially missing higher order
uncertainties (MHOUs) from using fixed order perturbative calculations as part
of the forward map. MHOUs have been included in parton distribution fits
\cite{AbdulKhalek:2019ihb} and in the future these should be included in the
closure test, however this is beyond the scope of the study presented here,
since MHOUs are still not included in the NNPDF methodology by default. In the
results presented in the rest of this paper we do include nuclear and deuteron
uncertainties, as presented in \cite{Ball:2018twp, Ball:2020xqw}, since they are
to be included in NNPDF fits by default. Extensive details for including
theoretical uncertainties, modelled as theoretical covariance matrices can be
found in those references. For the purpose of this study the modelisation
uncertainty is absorbed into the prior of the data, since
\begin{equation}
    \obspriorcent = \fwdobsop(\modelvec) + \obsnoise + \delta
\end{equation}
where $\delta \sim \mathcal{N}(0, \cov^{\rm theory})$. But since the
modelisation uncertainty is independent of the data uncertainty then from the
point of view of a closure test we can absorb $\delta$ into $\obsnoise$ by
modifying the data prior: $\obsnoise \sim \mathcal{N}(0, C + C^{\rm theory})$,
we must also update the likelihood of the data given the model to use the total
covariance $(C + C^{\rm theory})$. From now onwards we will omit $C^{\rm
theory}$ because it is implicit that we always sample and fit data using the
total covariance matrix which includes any modelisation uncertainty we currently
take into account as part of our methodology.

\section{Data space estimators} \label{sec:ClosureEstimators}

\subsection{Deriving the data space estimators}
\label{sec:ClosureEstimatorsDerivation}

We start by defining the model error as the expectation across models of the
$\chi^2$ between the model predictions and some data central values
$\testset{\obspriorcent}$, normalised by the number of data points
\begin{equation}
    \label{eq:chi2kerep}
    \eout = \frac{1}{\ndata} \emodel{
        \left( \testset{\fwdobsop}\left(\modelvecrep\right) - \testset{\obspriorcent} \right)^T
        \testset{\obspriorcov}^{-1}
        \left( \testset{\fwdobsop}\left(\modelvecrep\right) - \testset{\obspriorcent} \right)
    }\, ,
\end{equation}
where we defined the expectation value over the ensemble of model replicas as
\begin{equation}
    \emodel{x} \equiv \frac{1}{\nreps} \sum_{k=1}^{\nreps} x \, ,
\end{equation}
and we purposely denoted the data which the model error is evaluated on as
$\testset{\obspriorcent}$, as opposed to the data which is used to determine the model
parameters $\obspriorcent$. We could of course set
$\testset{\obspriorcent} = \obspriorcent$ and
evaluate the model performance on the fitted data however, as is common in
machine learning literature, we intend to use a separate set of test data.
Ideally we would choose $\testset{\obspriorcent}$ such that $\testset{\obspriorcent}$
and $\obspriorcent$ are statistically independent, as in Eq.~\ref{eq:JointIndepDataPrior}.
This is achieved by choosing the split such that the experimental covariance matrix
is block diagonal:
\begin{equation}
    \modelpriorcov^{\rm Total} =
    \begin{bmatrix}
        \modelpriorcov  & 0  \\ 
        0  & \testset{\modelpriorcov}  \\ 
    \end{bmatrix}
\end{equation}
In the
context of a closure test, $\eout$ is a stochastic quantity which depends both
on the training data, through the ensemble of MAP estimators, and the test data.

It is useful perform a decomposition of this expression, a similar exercise is
performed for the likelihood function associate with least-squares regression in
\cite{mlforphysics}. Least-squares regression is a special case of minimum
likelihood estimation, where the uncertainty on each data point is equal in
magnitude and uncorrelated. Here we review the decomposition in the more general
framework of data whose uncertainty is multigaussian. Starting with
Eq.~\ref{eq:chi2kerep} (evaluated on the ideal test data), we can complete the
square
\begin{equation}
    \begin{split}
    \label{eq:EoutDecomposition}
        &\eout = \frac{1}{\ndata} \emodel{
            \left( \testset{\fwdobsop}\left(\modelvecrep\right) - \testset{\law} \right)^T
            \testset{\obspriorcov}^{-1}
            \left( \testset{\fwdobsop}\left(\modelvecrep\right) - \testset{\law} \right)
        } + \\
        &+ \emodel{
            \left( \testset{\law} - \testset{\obspriorcent} \right)^T
            \testset{\obspriorcov}^{-1}
            \left( \testset{\law} - \testset{\obspriorcent} \right)
        }+ \\
        &+ 2 \emodel{
            \left( \testset{\fwdobsop}\left(\modelvecrep\right) - \testset{\law} \right)^T
            \testset{\obspriorcov}^{-1}
            \left(\testset{\law} - \testset{\obspriorcent} \right)
        }\, .
    \end{split}
\end{equation}
The second term is the shift associated with evaluating the model error on
noisey test data and the final term is a cross term which we will deal with
later. We can decompose the first term further,
% TODO: add the same thing but for fully in sample data to an appendix.
\begin{equation}
    \begin{split}
        &\emodel{
            \left( \testset{\fwdobsop}\left(\modelvecrep\right) - \testset{\law} \right)^T
            \testset{\obspriorcov}^{-1}
            \left( \testset{\fwdobsop}\left(\modelvecrep\right) - \testset{\law} \right)
        } = \\
        &= \emodel{
            \left( \testset{\fwdobsop}\left(\modelvecrep\right) - 
            \emodel{\testset{\fwdobsop}\left(\modelvecrep\right)} \right)^T
            \testset{\obspriorcov}^{-1}
            \left( \testset{\fwdobsop}\left(\modelvecrep\right) - 
            \emodel{\testset{\fwdobsop}\left(\modelvecrep\right)} \right)
        } + \\
        &+ \left( \emodel{\testset{\fwdobsop}\left(\modelvecrep\right)} - \testset{\law} \right)^T
        \testset{\obspriorcov}^{-1}
        \left( \emodel{\testset{\fwdobsop}\left(\modelvecrep\right)} - \testset{\law} \right)\, ,
    \end{split}
\end{equation}
where we have used the fact that the second term is constant across replicas and
the cross term that arises in this decomposition is zero when the expectation
value across replicas is taken. The first term in this expression we call the
{\em variance} and the second term is the {\em bias}.

As previously mentioned $\eout$ should be considered a stochastic estimator, in
theory we could take the expectation value across training data $\obspriorcent$ and
test data $\testset{\obspriorcent}$, the latter of which cancels the cross term in
Eq.~\ref{eq:EoutDecomposition}. The final result of that would be
\begin{equation}\label{eq:ExpectedBiasVariance}
    \mathbf{E}_{\obspriorcent, \testset{\obspriorcent}}[\eout] =
    \mathbf{E}_{\obspriorcent}[{\rm bias}] + 
    \mathbf{E}_{\obspriorcent}[{\rm variance}] +
    \mathbf{E}_{\testset{\obspriorcent}}[{\rm noise}]\, .
\end{equation}
We are not interested in the observational noise term, since it is
independent of the model and in the limit of infinite test data
$\mathbf{E}_{\testset{\obspriorcent}}[{\rm noise}] \to 1$.
The two estimators of interest are independent of
the test data, and therefore we only need to take the expectation value over
the training data.

\paragraph{Multiple closure fits}
In practical terms, taking the expectation value across the training data can
be achieved by running multiple closure fits, each with a different
observational noise vector $\obsnoise$, and taking the average i.e.
\begin{equation}
    \mathbf{E}_{\obspriorcent}[ x ] = \frac{1}{\nfits} \sum_{j=1}^{\nfits} x.
\end{equation}
Clearly this is resource intensive, and requires us to perform many fits. In
NNPDF3.0 \cite{nnpdf30}, single replica proxy fits were used to perform a study
of the uncertainties. Here we have expanded the data-space estimators used in
the closure fits and also will be using multiple full replica fits to
calculate various expectation values - made possible by our next generation
fitting code.

\subsection{Geometric Interpretation}

It is possible to interpret the relevant data space estimators geometrically, by
considering a coordinate system where each basis vector corresponds to an
eigenvector of the experimental covariance matrix normalised by the square root
of the corresponding eigenvalue. The origin of the coordinate system is the true
value of the observable. The model predictions are then a set of points, where
the mean squared radius of those points is what we call the variance. The bias
is the l2-norm of the vector between the origin and the mean of the model
predictions. An example of this is given in Fig.~\ref{fig:diagram2destimators},
where for simplicity we have considered a system with just two data points, \ie\
a two-dimensional data space, with a diagonal covariance.
%
\begin{figure}
    \centering
    \includegraphics[width=0.8\textwidth]{diagonal_basis_2d_estimators_diagram.png}
    \caption{Example of geometric interpretation of closure test estimators.
    The origin
    is the true observable values for each data point. The level one data (or
    experimental central values) are
    shifted away from this by $\obsnoise$. In this example the covariance matrix
    is diagonal, so the eigenvectors correspond to the two data points, the
    square root of the eigenvalues are simply the standard deviation of those
    points. This is without loss of generality because any multivariate distribution
    can be rotated into a basis which diagonalises the covariance matrix.
    The 1-sigma observational noise confidence interval
    is a unit circle centered on the origin. Some closure
    estimators can be understood as l2-norms of the vectors connecting points,
    i.e the bias is the l2-norm of the vector from the origin to the central
    value of the predictions.}
    \label{fig:diagram2destimators}
\end{figure}
%

\subsection{Faithful uncertainties in data space}

The two closure estimators of interest, bias and variance, can be used to
understand faithful uncertainties in a practical sense. If we return to
Eq.~\ref{eq:ExpectedBiasVariance} we can examine both estimators in a bit more
detail.

\paragraph{Variance}

The {\em variance} in the above decomposition refers to the variance of the
model predictions in units of the covariance
\begin{equation}
    \label{eq:VarDef}
    \begin{split}
        % NOTE: not using \emodel here in order to split line.
        \var &= \frac{1}{\ndata}\mathbf{E}_{\{ \modelvec_* \}} \Big[ \\
            &\left( \testset{\fwdobsop}\left(\modelvecrep\right) - 
            \emodel{\testset{\fwdobsop}\left(\modelvecrep\right)} \right)^T
            \testset{\obspriorcov}^{-1}
            \left( \testset{\fwdobsop}\left(\modelvecrep\right) - 
            \emodel{\testset{\fwdobsop}\left(\modelvecrep\right)} \right)
        \Big] \, ,
    \end{split}
\end{equation}
which can be interpreted as the model uncertainty in the space of the test data.
It is instructive to rephrase Eq.~\ref{eq:VarDef} as
\begin{equation}
    \label{eq:VarDefalternative}
    \var = \frac{1}{\ndata} {\rm Tr} \left[ \covrep \testset{\obspriorcov}^{-1} \right],
\end{equation}
where 
\begin{equation}
    \label{eq:CovRep}
    \covrep = 
    \emodel{
            \left( \testset{\fwdobsop}\left(\modelvecrep\right) - 
            \emodel{\testset{\fwdobsop}\left(\modelvecrep\right)} \right)
            \left( \testset{\fwdobsop}\left(\modelvecrep\right) - 
            \emodel{\testset{\fwdobsop}\left(\modelvecrep\right)} \right)^T
        }
\end{equation}
is the covariance matrix of the predictions from the model replicas. Note that
we can rotate to a basis where $\testset{\obspriorcov}$ is diagonal,
\begin{equation}
    \label{eq:InvCovPrimeDiag}
    \left(\testset{\obspriorcov}^{-1} \right)_{ij} = \frac{1}{\left(\testset{\sigma}_i\right)^2} 
    \delta_{ij}\, ,
\end{equation}
then we can rewrite Eq.~\ref{eq:VarDefalternative} as 
\begin{equation}
    \label{eq:VarianceInterpretation}
    \var = \frac{1}{\ndata}\, \sum_i \frac{\covrep_{ii}}{\left(\testset{\sigma}_i\right)^2}\, .
\end{equation}
The numerator in the right-hand side of the equation above is the variance of
the theoretical prediction obtained from the fitted replicas, while the
denominator is the experimental variance, the average is now taken over
eigenvectors of the experimental covariance matrix.

% The following needs to be reworded or removed, the use of prior and posterior
% here is confusing because the prior here is nothing to do with the prior
% of the training data but the posterior is only the posterior of the training
% data and instead is the prior of the model when considering the test data..

% If the replicas are sampled from the
% posterior distribution of model, the right-hand side of the
% equation is the average reduction in the variance of the observables between the
% prior distribution, dictated by the experimental covariance, and the posterior
% distribution. The average is computed over the space of test data, \ie\ data
% points that are not seen by the fit. 

\paragraph{Bias}

Similarly, the {\em bias}\ is defined as the difference between the expectation
value of the model predictions and the true observable values in units of the
covariance, \ie 
\begin{equation}
    \label{eq:BiasDef}
    \bias = \frac{1}{\ndata}
    \left( \emodel{\testset{\fwdobsop}\left(\modelvecrep\right)} - \testset{\law} \right)^T
    \testset{\obspriorcov}^{-1}
    \left( \emodel{\testset{\fwdobsop}\left(\modelvecrep\right)} - \testset{\law} \right)\, .
\end{equation}
The smaller the bias, the closer the central value of the predictions is to the
underlying law. In Eq.~\ref{eq:ExpectedBiasVariance}, the expectation value is
taken across the prior distribution of the training data, which yields
\begin{equation}
    \mathbf{E}_{\obspriorcent}[{\rm bias}] = \frac{1}{\ndata}
    {\rm Tr} \left[ \covcent \testset{\obspriorcov}^{-1} \right]\, ,
\end{equation}
where we have introduced $\covcent$ as the covariance of the expectation value
of the model predictions,
\begin{equation}
    \label{eq:CovCentDef}
    \covcent = 
    \mathbf{E}_{\obspriorcent}\left[
        \left( \emodel{\testset{\fwdobsop}\left(\modelvecrep\right)} - \testset{\law} \right)
        \left( \emodel{\testset{\fwdobsop}\left(\modelvecrep\right)} - \testset{\law} \right)^T   
    \right]\, .
\end{equation}
The point is that the bias on the test data is a stochastic variable which
depends on the central value of the training data through $\modelvecrep$. The
matrix $\covcent$ describes the fluctuations of the central value of the model
prediction around the true observable values as we scan different realisations
of the training data. 

It is important to stress the difference between variance and bias. In the case
of the variance, we are looking at the fluctuations of the replicas around their
central value for fixed $\obspriorcent$. This is related the ensemble of model
replicas we provide as the end product of a fit and can be calculated when
we have one instance of $\obspriorcent$, provided by the experiments.
In the case of the bias we consider the flucutations of the central value over
replicas around the true theoretical prediction as the values of $\obspriorcent$
fluctuate around $\law$. This latter procedure is only possible in a
closure test, where the underlying true observable is known. The
bias as defined here yields an estimate of the fluctuations of the MAP estimator
if we could do multiple independent experiments.

\paragraph{Bias-variance ratio}

Finally, the {\em bias-variance ratio} is defined as
\begin{equation}
    \label{eq:RatioDef}
    \biasvarratio \equiv \sqrt{\frac{
        \mathbf{E}_{\obspriorcent}[ \bias ]}{
            \mathbf{E}_{\obspriorcent}[ \var ]}}\, ,
\end{equation}
where we have taken the square root, since bias and variance are both mean
squared quantities. The value of $\biasvarratio$ yields a measurement of how
much uncertainties are over or under estimated. If the uncertainties are
completely faithful, then $\biasvarratio = 1$. We note that the relationship
does not work both ways and $\biasvarratio = 1$ does not necessarily guarantee
that the uncertainty is faithful. We also note that $\biasvarratio$ is not
completely general: it is not a measure defined in model space and depends on
the choice of test data. Therefore it only gives {\em local} information on the
model uncertainties. If the distribution of the expectation value of model
predictions is gaussian centered on the true observable values, with covariance
$\covcent$ and the distribution of the model replicas is also gaussian, with
covariance $\covrep$ then model uncertainties are faithful if
\begin{equation}\label{eq:IdealRatioDef}
    \covcent {\covrep}^{-1} = 1.
\end{equation}
The difficulty with calculating Eq.~\ref{eq:IdealRatioDef} comes from the fact
that $\covrep$ is likely to have large correlations which would lead it to be
singular or ill-conditioned. As a result, any error estimating $\covrep$ from
finite number of replicas could lead to unstable results. $\biasvarratio$
overcomes this instability by taking the ratio of the average across test data
of these matrices, in units of the experimental covariance matrix. There may
still be large relative errors for smaller eigenvalues of $\covrep$, but these
should not lead to instabilities in $\biasvarratio$ unless they correspond to
directions with very low experimental uncertainty. As an extra precaution, we
shall estimate an uncertainty on $\biasvarratio$ by performing a bootstrap
sample on fits and replicas.

\paragraph{Quantile statistics}
\label{sec:QuantileStatistics}

When the closure test was first presented in \cite{nnpdf30}, there was an estimator
introduced in the space of PDFs which also aimed to estimate faithfulness of
PDF uncertainties using the combined assumption of Gaussian PDF uncertainties
and quantile statistics, called $\xi_{1\sigma}$. Here we can define an
analogous expression in the space of data,
\begin{equation}
    \label{eq:XiDataDef}
    \xisigdat{n} = 
        \frac{1}{\ndata} \sum_{i}^{\ndata} 
        \frac{1}{\nfits} \sum_{l}^{\nfits}
            I_{[-n \testset{\sigma}_i^{(l)}, n \testset{\sigma}_i^{(l)}]}
            \left( \emodel{\testset{\fwdobsop}_i}^{(l)} - \testset{\law}_i \right),
\end{equation}
where $\testset{\sigma}_i^{(l)} = \sqrt{\covrep_{ii}}$ is the standard deviation of the
theory predictions estimated from the replicas of fit $l$ and $I_{[a, b]}(x)$
is the indicator
function, which is one when $a \leq x \leq b$ and zero otherwise. In other
words, $\xisigdat{n}$ is counting how often the difference between the prediction
from the MAP estimator and the true observable value is within the $n\sigma$-confidence
interval of the replicas, assuming they're Gaussian. Since $\covrep$
is primarily driven by the replica
fluctuations, we assume that it is roughly constant across fits, or independent
upon the specific instance of observational noise. This allows us to write
$\xisigdat{n}$ for a specific data point in the limit of infinite fits, each to
a different instance of the data as
%
\begin{equation}
    \label{eq:XiIExpecVel}
        \xisigdati{n} =
            \int_{-\infty}^{\infty} I_{[-n \testset{\sigma}_i, n \testset{\sigma}_i]}\,
            \left( \emodel{\testset{\fwdobsop}_i}^{(l)} - \testset{\law}_i \right) 
            \rho(\obsnoise) \, 
            {\rm d}(\obsnoise) \, ,
\end{equation}
%
where $\emodel{\fwdobsop_i}^{(l)}$ has implicit conditional dependence on
$\obsnoise$.
If
the distribution of \linebreak $\emodel{\testset{\fwdobsop}_i}^{(l)} - \testset{\law}_i$
is Gaussian, centered on
zero, we can defined ${\testset{ \hat{\sigma} }}_i = \sqrt{\covcent_{ii}}$. In which case
\begin{equation}
    \label{eq:expectedxi}
    \xisigdati{n} =
    \erf \left( \frac{n \testset{\sigma}_i}{\testset{\modelstd}_i \sqrt{2}}\right),
\end{equation}
which is simply the standard result of integrating a gaussian over some finite
symmetric interval.

The analogy between $\biasvarratio$ and $\xisigdat{n}$ is clear, the ratios of
uncertainties are both attempts to quantify Eq.~\ref{eq:IdealRatioDef} whilst keeping
effects due to using finite statistics under control. Whilst with
$\biasvarratio$ we take the average over
test data before taking the ratio, $\xisigdat{n}$ instead takes the ratio
of the diagonal elements - ignoring correlations. Since the predictions
from the model will be compared with experimental central values, taking into
account experimental error, we find it more natural to calculate $\xisigdat{n}$
in the basis which diagonalises the experimental covariance of the test data as
in Eq.~\ref{eq:InvCovPrimeDiag}. If we assume that in this new basis, that
both $\frac{\covrep_{ii}}{\left(\sigma'_i\right)^2}$ and
$\frac{\covcent_{ii}}{\left(\sigma'_i\right)^2}$ are approximately constant
for all eigenvectors of the experimental covariance matrix, then we recover the
approximation
\begin{equation}\label{eq:CompareXiRatio}
    \xisigdat{n} \sim \erf \left( \frac{ n\biasvarratio}{\sqrt{2}} \right).
\end{equation}
Whilst it's clear Eq.~\ref{eq:CompareXiRatio} is reliant on a fair few
assumptions which may not hold, we will use the comparison of $\xisigdat{n}$ with
$\biasvarratio$ to consider how valid these assumptions may be.

%\section{Linear Forward Map}

\subsection{Closure estimators - Linear forward map}

Once again we return to the linear model framework set out in
Sec.~\ref{sec:fluct-fit-values}. We can perform an analytical closure
test in this framework, and check our
understanding of the closure estimators. Consider the true observable
values for the test data is given by
\begin{equation}\label{eq:LinearLawMap}
    \testset{\law} = \testset{\fwdobsop} \lawmodel
\end{equation}
where $\lawmodel \in \modelspace$, which means the number of parameters
in the underlying law is less than or equal to the number of parameters in the
model, $\nlaw \leq \nmodel$. Using the previous results from
Sec.~\ref{sec:fluct-fit-values}, we can write down the
difference between the true observables and the predictions from the MAP estimator
(or the expectation of the model predictions across model replicas - in the
linear model these are the same)
\begin{equation}
    \begin{split}
        \emodel{\testset{\fwdobsop}\left(\modelvecrep\right)} - \testset{\law} &=
        \testset{\linmap} (\lawmodel - \linmap^+ \obspriorcent ) \\
        &= \testset{\linmap} \linmap^+ \obsnoise \, ,
    \end{split}
\end{equation}
where we recall that $\linmap$ is the forward map to the training observables
and $\obspriorcent$ are
the corresponding training central values. Calculating the covariance across
training data of
$\emodel{\testset{\fwdobsop}\left(\modelvecrep\right)} - \testset{\law}$
gives
\begin{equation}
    \covcent = \testset{\linmap} \linmap^+ \obspriorcov (\testset{\linmap} \linmap^+)^T \, ,
\end{equation}
so the full expression for $\mathbf{E}_{\obspriorcent}[{\rm bias}]$ is given by
\begin{equation}
    \mathbf{E}_{\obspriorcent}[{\rm bias}] = \frac{1}{\ndata}
    {\rm Tr} \left[
        \testset{\linmap} \linmap^+ \obspriorcov (\testset{\linmap} \linmap^+)^T
        \testset{\modelpriorcov}^{-1}
    \right].
\end{equation}
We note that if the test data is identical the data the model was fitted on,
we recover an intuitive result $\mathbf{E}_{\obspriorcent}[{\rm bias}] = \frac{\nmodel}{\ndata}$.
Consider the example of the polynomial, the maximum value which $\nmodel$ can
take whilst $\linmap$ still has linearly independent rows is $\ndata$ and in this case
the $\mathbf{E}_{\obspriorcent}[{\rm bias}]$ takes it's maximum value of 1. The central
predictions from the model exactly pass through each data point.

We can perform a similar exercise on the model replica predictions. The difference
between the predictions from model replica $\repind$ and the expectation value
of the model predictions is
\begin{equation}
    \begin{split}
        \testset{\fwdobsop}\left(\modelvecrep\right) -
        \emodel{\testset{\fwdobsop}\left(\modelvecrep\right)} &=
        \testset{\linmap} (\linmap^+ \pseudodat^{\repind} - \linmap^+ \modelpriorcent) \\
        &= \testset{\linmap} \linmap^+ \noise^{\repind}.
    \end{split}
\end{equation}
Since $\noise$ and $\obsnoise$ follow the same distribution, it's clear that
\begin{equation}
    \covrep = \covcent,
\end{equation}
which, as a result means that
\begin{equation}
    \var = \mathbf{E}_{\obspriorcent}[{\rm bias}].
\end{equation}
We recall that when the map is linear, the NNPDF MC methodology generates replicas
which are sampled from the posterior distribution of the model given the data.
We have shown here that provided the underlying law belongs to the model
space, the posterior distribution of the model predictions satisfy the
requirement that $\biasvarratio = 1$.

\subsection{Relating variance to posterior distribution}

If the training and test split is ideal, as discussed in
Sec.~\ref{sec:ClosureEstimatorsDerivation}, such that the two sets of data
are uncorrelated then it's
possible to perform a sequential marginalisation of the data \cite{tarantola}.
The posterior model distribution obtained by marginalising over the training
data, can be used as a prior when viewing the test data from a Bayesian
perspective. The model prior for the test data is given by
Eq.~\ref{eq:PosteriorMultiGaussModel}, which is fully
parameterised by mean and covariance: $\modelpriorcent = \linmap^+ \obspriorcent$ and
$\modelpriorcov = \linmap^+ \obspriorcov (\linmap^+)^T$ respectively.
The posterior data-space distribution
of the new data is given by Eq.~\ref{eq:PosteriorDataSpace}, which is also a
Gaussian with mean and covariance given by Eq.~\ref{eq:PosteriorDataParams}.
Now if we take $\testset{\obspriorcov} \to \infty$, which is equivalent to
there being prior information for the test data, the posterior data-space distribution
is given by
\begin{equation}
    \begin{split}
        \pi_D(\testset{\obspriorcent} |\obspriorcent, \modelpriorcent, \linmap)
        &\propto \exp \left( - \frac{1}{2}
            (\testset{\obspriorcent} - \testset{\linmap} \linmap^+ \modelpriorcent)^T
            {{\testset{\linmap}}^T}^+ \linmap^T \obspriorcov \linmap {\testset{\linmap}}^+
            (\testset{\obspriorcent} - \testset{\linmap} \linmap^+ \modelpriorcent)
        \right) \, ,
    \end{split}
\end{equation}
a Gaussian with covariance
$\left( {{\testset{\linmap}}^T}^+ \linmap^T \obspriorcov \linmap {\testset{\linmap}}^+ \right)^{-1}$
and mean $\testset{\linmap} \linmap^+ \modelpriorcent$. We recognise that the
covariance of posterior
distribution of the test data, is the covariance of model predictions for the
test data, $\covrep$. Taking $\testset{\obspriorcov} \to \infty$ might seem
slightly peculiar, however we can understand this as treating the test dataset
as new observables for which there are no experimental measurements. When comparing
$\covcent$, the covariance of the MAP estimators, to $\covrep$ we are
checking that the observed fluctuation of the MAP estimators about the true
observables is consistent with the posterior data-space covariance.

\subsection{Underparameterised model}

Note that if we were to choose the
number of model parameters such that $\nlaw > \nmodel$, then the variance
would be unaffected, since the underlying law parameters cancel. However, the
bias would now contain an extra term from the extra parameters in the
underlying law, schematically:
\begin{equation}
    \begin{split}
        (\emodel{\testset{\fwdobsop}\left(\modelvecrep\right)} - \testset{\law})_i =
        \sum_{1 \leq j \leq \nmodel} \testset{\linmap}_{ij} (\linmap^+ \obsnoise)_j +
        \sum_{\nmodel < j \leq \nlaw} \testset{\linmap}_{ij} \lawmodel_j,
    \end{split}
\end{equation}
which would mean that $\biasvarratio \neq 1$. This demonstrates that requiring
$\biasvarratio = 1$ demands that the model space is suitably flexible, if the
underlying law is parameterised then this can be summarised as requiring
$\lawmodel \in \modelspace$. Note that in the
underparameterised regime the model replicas are still drawn from the posterior
distribution, however because $\lawmodel \notin \modelspace$ we've somehow
invalidated the assumptions that go into the relation between model predictions
and the data-space prior.

Although $\biasvarratio$ was largely chosen on practical
grounds, we see that it is still a stringent test that our assumptions are
correct and that the distribution our model replicas are drawn from is meaningful,
this is what we mean when we say {\em faithful uncertainties}.

An unfortunate
trade-off when using $\biasvarratio$ is it can't be used as a diagnostic
tool, and is instead used simply for validation. For example, if
$\biasvarratio > 1$, then we
can't know whether there was a problem with the fitting methodology used to
generate the model replicas or a deeper issue such as an inflexible model.

\section{Neural network parton distribution functions}
% TODO: here we should properly define the PDFs, forward map and which experimental
% data we fitted and which was used in the test set.

When fitting experimental data we vary the parameters of a set of PDF replicas
at the initial scale such that the $\chi^2$ is minimised between the
corresponding theory predictions and a generated pseudodata replica. A set of
PDFs usually refers to a set of seperate continuous functions, one for each
flavour of PDF in a particular basis. In this specific study, fits performed
parameterise the set of PDFs as a single neural network which takes
as input $x$ and $\ln x$ and returns 8 outputs, one for each flavour in the
fitting basis, multiplied by some preproccessing exponents. The output for a
single flavour $j$ is
\begin{equation}
    NN(x, \ln x)_j * x^{1-\alpha_j} * (1-x)^{\beta_j},
\end{equation}
where each flavour has it's own preproccessing exponents $\alpha$ and $\beta$,
parameters that are varied in these fits, and $NN(x, \ln x)_j$ is the
$j^{\rm th}$ output from the neural network.
When an experiment is included in an NNPDF fit, we take the published
experimental central values and uncertainties (statistical and systematic)
and use these pieces of information to generate the pseudodata.
The pseudodata replica is generated
through Monte Carlo sampling by applying noise to the experimental
central values.
After fitting many sets of PDF replicas (usually of order 100 sets),
each set to an independently generated pseudodata replica, we have an ensemble of
PDF replicas.

%\section{Experimental setup}

Here we discuss the experimental setup used to produce the results. The results
here act both as a proof of principle of the data space estimators presented in
this paper but also as part of a suite of methodological validation tools, see
also the "future tests" \cite{Cruz_Martinez_2021}, used to understand the PDF
uncertainties of the upcoming NNPDF4.0 set of PDFs. For the purpose of
understanding how the results here were produced, we will briefly describe the
key features of the NNPDF4.0 methodology, but refer the reader to NNPDF4.0 for a
full discussion on how these methodological choices were made, and the impact on
performing PDF fits to experimental data.

\subsection{Neural network parton distribution functions}

Using neural networks to fit PDFs has been discussed many times in previous
NNPDF publications, see for example \cite{nnpdf30, Ball_2017}. A new feature of
NNPDF4.0 will be that, for the default fit performed in the evolution basis, a
single neural network parameterises all 8 PDF flavours $\{ g, \Sigma, V, V_3,
V_8, T_3, T_8, c^+ \}$ at the initial scale. The PDF for a single flavour $j$,
at the initial scale $Q_0 = 1.65~{\rm GeV}$ is given by
\begin{equation}
    f_j(x, Q_0) = NN(x, \ln x | \modelvec)_j * x^{1-\alpha_j} * (1-x)^{\beta_j},
\end{equation}
where $\alpha$ and $\beta$ are the preprocessing exponents, which control the
PDF behaviour at $x \to 0$ and $x \to 1$ respectively and $NN(x, \ln x |
\modelvec)_j$ is the $j^{\rm th}$ output of the neural network, which takes $x$
and $\ln x$ as input. As discussed in Sec.~\ref{sec:fit-reps}, an ensemble of
models is fitted, each one is an MAP estimator of the corresponding pseudo-data
it is fitted on. Unlike in the case of the linear model, the parameters of the
neural network cannot be found analytically and instead an optimization
algorithm is used to try and find the parameters which maximise the likelihood.
In principle, the preprocessing exponents can also be varied during the fit
analogously to the neural network parameters, such as in \cite{Carrazza_2019},
or they can be randomly selected from a predetermined range as is done in
previous NNPDF releases, for example \cite{Ball_2017}. There are clearly many
choices with respect to hyperparameters, the discussion of how these choices
have been made is beyond the scope of this paper and left to the full NNPDF4.0
release \cite{NNPDF40}. A summary of the hyperparameters used to produce results
presented in this paper are provided in Tab.~\ref{tab:Hyperparams}.

Finally, the parton distributions themselves are not compared directly with
data. Instead the observables quantities are obtained by performing convolutions
with the PDFs, as discussed in \ref{eq:DISExample}. In practice the observables
are obtained by convoluting the PDFs with FastKernel tables, presented in
\cite{Ball_2010,Bertone_2017}, for each data point. The convolution depends on
the process type of the observable, for DIS-like observables, such as in
Eq.~\ref{eq:DISExample}, the convolution is performed with a single PDF. For
hadronic observables the convolution is performed between two PDFs.

\subsection{Closure test setup}

As input to the closure test, a single replica was drawn randomly from a
previous NNPDF fit to experimental data. We refer to this as the underlying law
and the corresponding predictions the true observable values. An example of the
gluon input is provided in Fig.~\ref{fig:InputGluonPDF}. In principle any
function could be used as underlying law, however it makes sense to use a
realistic input.

\begin{figure}
    \centering
    \includegraphics[width=0.8 \textwidth]{plot_pdfs_g.png}
    \caption{The green line is the input underlying law for the gluon PDF,
    which is sampled from the ensemble from a fit to data. The 68\% confidence
    interval is plotted for those replicas as the orange band.}
    \label{fig:InputGluonPDF}
\end{figure}

The observables used in the fits are a subset of the full NNPDF4.0 dataset. For
convenience, we chose to fit the PDFs on a variant of the NNPDF3.1 dataset used
in Ref.~\cite{Ball_2018}, which is described in detail in a study of the
determination of the strange PDF~\cite{Faura_2020}. The datasets used in the
calculation of statistical estimators are the new datasets which will be
included in NNPDF4.0, which will be discussed in detail with the main release.
For a full summary of observables used in the test data and a visual
representation of the kinematic region of both the training and testing data,
see App.~\ref{sec:appendix-datasets}.

The partitioning of the available data into fitting and testing should not
affect the interpretation of the closure test results. However, one
could consider splitting the data into training and test in a way which was
physically motivated \eg the partitioning could have been loosely
stratified on the kinematic coverage of the data rather than this naive
chronological splitting.
Alternatively, since the data is generated from the theory predictions produced
by the input underlying law, one could even produce completely artificial data
using a different set of FK tables. From a practical standpoint, using the
NNPDF3.1 dataset and validating on the newly included datasets in 4.0 allowed us
to validate the PDF uncertainities on data outside of the kinematic coverage of
data included in the fit. Furthermore, the data estimators only give us local
information on the PDF uncertainties and it seems logical to split the data in
this way since the results seem more applicable to the reality of how the PDFs
end up being used.

We then generate 30 different sets of experimental central values (or L1 data),
as discussed in Sec.~\ref{sec:closure-test-intro}, for the fitted 3.1-like
dataset. Each set of experimental central values was then fitted following
NNPDF4.0 methodology \cite{NNPDF40}, producing 40 pseudo-data replicas.
\section{Results}
\paragraph{Note}{Do we want results in and out of sample? Still don't
fully understand in sample results. Need to add finite replica/fit results}

Also need to define list of datasets and new vs old processes

\subsection{Data space estimators}

\paragraph{Note}{Add bias and variance across fits to show respective spreads}

\paragraph{Note}{Add xi across datapoints for each experiment}

First we show $\sqrt{\frac{\eshift{\bias}}{\eshift{\var}}}$ by experiment for data
not included in the fit, but old processes.

\begin{center}
    \begin{tabular}{lrr}
        \toprule
        {} &  $\ndata$ &  $\sqrt{\frac{\eshift{\bias}}{\eshift{\var}}}$ \\
        experiment &        &                      \\
        \midrule
        HERACOMB   &     63 &                 1.60 \\
        ATLAS      &    371 &                 0.76 \\
        CMS        &    310 &                 0.87 \\
        LHCb       &     35 &                 0.83 \\
        Total      &    779 &                 0.89 \\
        \bottomrule
        \end{tabular}
\end{center}

Next we show the same estimator, $\sqrt{\frac{\eshift{\bias}}{\eshift{\var}}}$, by
experiment for data not included in the fit, new processes

\begin{center}
    \begin{tabular}{lrr}
        \toprule
        {} &  $\ndata$ &  $\sqrt{\frac{\eshift{\bias}}{\eshift{\var}}}$ \\
        experiment &        &                      \\
        \midrule
        ATLAS      &    374 &                 1.00 \\
        CMS        &    179 &                 0.81 \\
        Total      &    553 &                 0.90 \\
        \bottomrule
        \end{tabular}
\end{center}

Now we compare the measured $\xi_{1\sigma}$ and estimated $\xi_{1\sigma}$ by experiment
for data not included in the fit, old processes. To estimate $\xi_{1\sigma}$
we we take $\sqrt{\frac{\eshift{\bias}}{\eshift{\var}}}$ and substitute into
eq. \eqref{eq:expectedxi}, assuming that $\frac{\sigma_i}{\modelstd_i}$ is
constant across datapoints and equal to $\sqrt{\frac{\eshift{\bias}}{\eshift{\var}}}$.

\begin{center}
    \begin{tabular}{lrrr}
        \toprule
        {} &  $\ndata$ &  measured $\xi_{1\sigma}$ &  estimated $\xi_{1\sigma}$ from bias/variance \\
        experiment &        &                           &                                                  \\
        \midrule
        HERACOMB   &     63 &                      0.46 &                                             0.47 \\
        ATLAS      &    371 &                      0.77 &                                             0.81 \\
        CMS        &    310 &                      0.69 &                                             0.75 \\
        LHCb       &     35 &                      0.78 &                                             0.77 \\
        Total      &    779 &                      0.72 &                                             0.74 \\
        \bottomrule
        \end{tabular}
\end{center}


Similarly, we compare the same estimators $\xi_{1\sigma}$ and estimated
$\xi_{1\sigma}$ by experiment for data not included in the fit, new processes.

\begin{center}
    \begin{tabular}{lrrr}
        \toprule
        {} &  $\ndata$ &  measured $\xi_{1\sigma}$ &  estimated $\xi_{1\sigma}$ from bias/variance \\
        experiment &        &                           &                                                  \\
        \midrule
        ATLAS      &    374 &                      0.66 &                                             0.69 \\
        CMS        &    179 &                      0.76 &                                             0.78 \\
        Total      &    553 &                      0.69 &                                             0.73 \\
        \bottomrule
        \end{tabular}
\end{center}


\subsection{PDF space estimators}

TODO.

\section{Summary}

We've presented a formal framework for inverse problems, from a Bayesian
perspective. In particular, the framework provides a more formal understanding
of what it means when we talk about propagating experimental uncertainities
into the space of models. The framework itself doesn't concern itself with
the concept of fitting parameters, and the posterior model distribution is simply
the result of marginalising the joint prior distribution of the model and
data. We note that sampling from the posterior distribution of the model is,
in general, highly non-trivial but show that at least for linear problems the
NNPDF MC methodology can be shown to produce a sample of models which are
distributed according to the posterior model distribution. Furthermore, we
provide evidence that even for non-linear models this result at least holds
as a good approximation close to the MAP estimator.

We then use this formal framework to think about some of the estimators which
we use as part of the NNPDF closure test. In particular we derive bias and
variance from decomposing an out of sample error function, but then show
that these estimators can be related back to the posterior distributions in
the Bayesian framework. We note that the estimators themselves are not perfect
and suffer from only testing the model uncertainties locally (in regions where
the test data probes). Furthermore, the estimators only give an approximate
overall picture, and cannot be used to diagnose where the problem arises if
we find evidence that the model uncertaintes are not faithful.

Given the framework set out here, future work should be undertaken to try and
design some more general closure estimators in model space. Combining the ideas
of the data space closure estimators with the formal understanding of
infinite-dimensional probability measures.

We give some preliminary closure results, as a proof of principle, the results
presented here will be examined in more detail alongside the full NNPDF4.0
release but serve as an example of how the data space estimators can be practically
included even in a rather complex setting. The preliminary results are very
positive and provide evidence that for unseen data, the NNPDF methodology
appears to provide faithful uncertainities. As previously mentioned, a more
general set of estimators in model space would be the gold-standard and give
us confidence in our uncertainities for future observables which probe
regions which are not covered by either the training or testing data.

Something which was touched upon, but not investigated in detail, is that a
closure test has fully consistent observable central values and uncertainities
by construction which is likely not the case in real world fits. Other
future work could focus on whether a closure test could be designed where this
is not the case, and what faithful uncertainities would mean in this context.
There already exist methodologies which deal with inconsistent data (or unknown
systematics) in a Bayesian framework,
for example in these cosmological studies \cite{Luis_Bernal_2018, Hobson_2002},
and so the framework we presented here facilitates using this kind of
formal approach to including inconsistent data. This should emphasise the
importance of future developments to the NNPDF methodology keeping in touch
with the formal understanding of inverse problems, since we can draw on a wide
range of techniques from other disciplines who are also using a Bayesian
approach.

As a final thought, in light of the Bayesian framework set out here, one could
even conceive of a methodology which
used a different MC technique to sample directly from the posterior model
distribution, for which we have an explicit (unnormalised) expression. This
would be a complete paradigm shift from the approach described in
Sec.~\ref{sec:fluct-fit-values} but could have some advantages, such as
guaranteeing that the model replicas were exactly sampled from the posterior
model distribution, even in the tails of the distribution far from the MAP
estimator.

\appendix
\section{Gaussian integrals}
\label{sec:GaussianIntegrals}

Theory errors can be included in this framework by allowing the distribution of
observables around the theory prediction to have a finite width, \eg\ by
replacing the Dirac delta 
\begin{equation}
    \label{eq:DiracInApp}
    \delta(y-\mathcal{G}u)
\end{equation}
in Eq.~\ref{eq:ThetaCorr} with a Gaussian
\begin{equation}
    \label{eq:TheoryGaussian}
    \theta(\obs,\modelvec|\fwdobsop) \propto \exp\left[
        -\frac12 \left(y-\mathcal{G}u\right)^T
        C_T^{-1} \left(y-\mathcal{G}u\right)
    \right]\, .
\end{equation}
For the purposes of this study, we do not want to provide a realistic estimate
of theory errors. Instead we will be assuming that the errors are uncorrelated
and identical for all data points
\begin{equation}
    \label{eq:DiagTheoryCov}
    C_T = \sigma^2 \mathds{1}\, ,
\end{equation}
and we will be interested in the limit where $\sigma^2\to 0$. 

\subsection{Integrating out the data}
\label{sec:IntOutData}

Marginalizing with respect to \obs\ in this case yields 
\begin{align}
  \label{eq:MarginGaussData}
  \pi_M(\modelvec|\obspriorcent,\modelpriorcent,\fwdobsop) 
  &\propto \pi_{M}^0(\modelvec|\modelpriorcent) \, 
  \int dy\, \pi_{D}^0(\obs|\obspriorcent) 
    \theta(\obs,\modelvec|\fwdobsop) \, .
\end{align}
The argument of the exponential in the integrand is a quadratic form in \obs, 
\begin{equation}
    \label{eq:QuadFormDataInt}
    A = \left(y-y_0\right)^T C_D^{-1} \left(y-y_0\right) +
    \left(y-\mathcal{G}u\right)^T C_T^{-1} \left(y-\mathcal{G}u\right)\, .
\end{equation}
The integral can be easily evaluated by completing the square, 
\begin{equation}
    \label{eq:QuadFormDataIntSquare}
    A = \left(y-\tilde{y}\right)^T 
    \tilde{C}_D^{-1}
    \left(y-\tilde{y}\right) + R_D\, .
\end{equation}
Comparing Eqs.~\ref{eq:QuadFormDataInt} and~\ref{eq:QuadFormDataIntSquare} at order $y^2$ and \obs, yields
\begin{align}
    \tilde{C}_D^{-1} &= \frac{1}{\sigma^2}
    \left(\mathds{1} + \sigma^2 C_D^{-1}\right)\, , \\
    \tilde{y} &= \left(\mathds{1} + \sigma^2 C_D^{-1}\right)^{-1} 
    \left(
        \mathcal{G}u + \sigma^2 C_D^{-1} y_0
    \right)\, ,
\end{align}
and therefore
\begin{align}
    \tilde{y}^T \tilde{C}_D^{-1} \tilde{y}
    &= \frac{1}{\sigma^2} \left(\mathcal{G}u\right)^T
    \left(\mathds{1}+\sigma^2 C_D^{-1}\right)^{-1} \left(\mathcal{G}u\right) +
    y_0^T C_D^{-1} \left(\mathds{1}+\sigma^2 C_D^{-1}\right)^{-1} 
    \left(\mathcal{G}u\right) + \nonumber \\
    \label{eq:RemainderFromSquare}
    & \quad + \left(\mathcal{G}u\right)^T C_D^{-1} 
    \left(\mathds{1}+\sigma^2 C_D^{-1}\right)^{-1} y_0 + 
    \sigma^2 y_0^T C_D^{-1} \left(\mathds{1}+\sigma^2 C_D^{-1}\right)^{-1} 
    C_D^{-1} y_0\, .
\end{align}
Note that the four terms in the equation above are ordered in increasing powers
of $\sigma^2$ and ultimately we will be interested in the limit $\sigma^2\to 0$,
which reproduces the Dirac delta in $\theta(y,u)$. Plugging
Eq.~\ref{eq:RemainderFromSquare} in Eq.~\ref{eq:QuadFormDataIntSquare} and again
comparing to Eq.~\ref{eq:QuadFormDataInt}, we find
\begin{equation}
    \label{eq:RDBeforeLimit}  
    \begin{split}
    R_D 
    &= \frac{1}{\sigma^2} \left(\mathcal{G}u\right)^T 
    \left[
        \mathds{1} - \frac{1}{\mathds{1}+\sigma^2 C_D^{-1}} 
    \right]
    \left(\mathcal{G}u\right) - y_0^T C_D^{-1} \left(\mathcal{G}u\right)
    - \left(\mathcal{G}u\right)^T C_D^{-1} y_0 + \\ 
    & \quad + y_0^T C_D^{-1} y_0 + \mathcal{O}(\sigma^2)\, ,       
    \end{split} 
\end{equation}
Expanding for small values of $\sigma^2$ the terms of order $1/\sigma^2$ cancel;
keeping only finite terms in the limit $\sigma^2 \to 0$ we finally obtain
\begin{equation}
    \label{eq:RDAfterLimit}
    R_D = \left(\mathcal{G}u - y_0\right)^T C_D^{-1}
    \left(\mathcal{G}u - y_0\right)\, .
\end{equation}
This is exactly the result that we obtained earlier when 
\begin{equation}
    \label{eq:RemindTheta}
    \theta(y,u|\mathcal{G}) = \delta(y-\mathcal{G}u)\, .
\end{equation}
It should not come as a surprise since in the limit where $\sigma^2 \to 0$ the
Gaussian distribution that we chose to describe the fluctutations of the data
around the theory predictions reduces indeed to a Dirac delta. The posterior for
the model is exactly the one we computed in Sect.~\ref{sec:inverse-problems}. We
do not learn anything new from this exercise, but it is a useful warm-up for the
next example. The integral over \obs\ can now be performed easily, since it is
yet again a Gaussian integral. 


\subsection{Integrating out the model}
\label{eq:IntModOut}

Using the same approach as above, we now want to marginalise with respect to the model in order to obtain the posterior distribution of the data: 
\begin{align}
    \label{eq:MarginGaussModel}
    \pi_D(\obs|\obspriorcent,\modelpriorcent,\fwdobsop) 
    &\propto \pi_{D}^0(\obs|\obspriorcent) \, 
    \int du\, \pi_{M}^0(\modelvec|\modelpriorcent) 
      \theta(\obs,\modelvec|\fwdobsop) \, .
  \end{align}
We follow exactly the same procedure outlined above, starting from the argument
of the exponential
\begin{equation}
    \label{eq:QuadFormModelInt}
    A = \left(u-u_0\right)^T C_M^{-1} \left(u-u_0\right) +
    \left(y-\mathcal{G}u\right)^T C_T^{-1} \left(y-\mathcal{G}u\right)\, ,
\end{equation}
we complete the square and rewrite it in the form
\begin{equation}
    \label{eq:QuadFormModelIntSquare}
    A = \left(u-\tilde{u}\right)^T 
    \tilde{C}_M^{-1}
    \left(u-\tilde{u}\right) + R_M\, .
\end{equation}
It can be readily checked that in this case
\begin{align}
    \tilde{C}_M^{-1} &= \frac{1}{\sigma^2}
    \left(\mathcal{G}^T \mathcal{G} + \sigma^2 C_D^{-1}\right)\, , \\
    \tilde{u} &= \left(\mathcal{G}^T \mathcal{G} + \sigma^2 C_M^{-1}\right)
    \left(
        \mathcal{G}^T y + \sigma^2 C_M^{-1} u_0
    \right)\, .
\end{align}
In order to evaluate $R_M$, we need
\begin{equation}
    \label{eq:UtildeUtildeTerm}
    \begin{split}
    \tilde{u}^T \tilde{C}_M^{-1} \tilde{u} 
    &= \frac{1}{\sigma^2} y^T \mathcal{G} 
    \left(\mathcal{G}^T \mathcal{G} + \sigma^2 C_M^{-1}\right)^{-1}
    \mathcal{G}^T y +
    u_0^T C_M^{-1} 
    \left(\mathcal{G}^T \mathcal{G} + \sigma^2 C_M^{-1}\right)^{-1}
    \mathcal{G}^T y + \\
    & \quad + y^T \mathcal{G} \left(\mathcal{G}^T \mathcal{G} + \sigma^2 C_M^{-1}\right)^{-1} C_M^{-1} u_0 + \mathcal{O}(\sigma^2)\, .
    \end{split} 
\end{equation}
Noting that
\begin{equation}
    \label{eq:InverseFromTarantola}
    \left(\mathcal{G}^T \mathcal{G} + \sigma^2 C_M^{-1}\right)^{-1} =
    \frac{1}{\sigma^2} C_M - 
    \frac{1}{\sigma^2} C_M \mathcal{G}^T 
    \left(\mathcal{G} \frac{1}{\sigma^2} C_M \mathcal{G}^T 
        + \mathds{1}\right)^{-1} \mathcal{G}
    \frac{1}{\sigma^2} C_M\, ,
\end{equation}
we have, in the limit where $\sigma^2 \to 0$
\begin{equation}
    \label{eq:QuadraticYTerm}
    \mathcal{G} \left(\mathcal{G}^T \mathcal{G} + \sigma^2 C_M^{-1}\right)^{-1} 
    \mathcal{G}^T = 
    \mathds{1} - \sigma^2 \left(\mathcal{G} C_M \mathcal{G}^T\right)^{-1} + 
    \mathcal{O}(\sigma^4)\, .
\end{equation}
Collecting all terms we find
\begin{equation}
    \label{eq:RMAfterLimit}
    R_M = \left(y - \mathcal{G} u_0\right)^T 
        \left(\mathcal{G} C_M \mathcal{G}^T\right)^{-1}
        \left(y- \mathcal{G} u_0\right)\, .
\end{equation}
Performing the Gaussian integral over $u$ in Eq.~\ref{eq:MarginGaussModel}, we
obtain the posterior distribution of the data
\begin{equation}
    \label{eq:PosteriorDataDistr}
    \pi_D^y(y) \propto 
    \exp\left[-\frac12 \left(y-y_0\right)^T C_D^{-1} \left(y - y_0\right)
    -\frac12 \left(y - \mathcal{G} u_0\right)^T 
    \left(\mathcal{G} C_M \mathcal{G}^T\right)^{-1}
    \left(y - \mathcal{G} u_0\right)
    \right]\, .
\end{equation}
As in the case above, we note that this is a Gaussian distribution,
\begin{equation}
    \label{eq:PosteriorDataDistrGauss}
    \pi_D^{y}(y) \propto
    \exp \left[
        -\frac12 \left(y - \tilde{y}\right)^T
        \tilde{C}_D^{-1} 
        \left(y - \tilde{y}\right)
    \right]\, ,
\end{equation}
where the mean and the covariance are given in
Eqs.~\ref{eq:PosteriorDataParamsMean} and~\ref{eq:PosteriorDataParamsCov}. We
can rewrite those expressions as
\begin{align}
    \tilde{y} &= \mathcal{G} \tilde{u} \, , \\
    \tilde{C}_D^{-1} &=
        C_D^{-1} + \left(\mathcal{G} C_M \mathcal{G}^T\right)^{-1}\, .
\end{align}
In order to simplify the notation we introduce
\begin{equation}
    \hat{C}_M = \left(\mathcal{G} C_M \mathcal{G}^T\right)\, ,
\end{equation}
and then
\begin{equation}
    \tilde{C}_D = \hat{C}_M
    - \hat{C}_M \left(\hat{C}_M + C_D \right)^{-1} 
    \hat{C}_M\, .
\end{equation}
\section{Closure test datasets}
\label{sec:appendix-datasets}

The full list of datasets included in the test set are shown in
Tab.~\ref{tab:summarise_new_data}. The central values are not actually used
in the closure test, however we use the experimental uncertainities in
the calculation of both $\biasvarratio$ and $\xisigdat{1}$. The corresponding
predictions generated from the underlying law are used as the true
observable values. Neither of the data-space closure estimators rely on
the central values of the test datasets.

\begin{table}[h!]
    \begin{center}
        \begin{tabular}{lc}
    \toprule
    Data set
    & Ref.
    \\
    \midrule
    DY E906 $\sigma^d_{\rm DY}/\sigma^p_{\rm DY}$ (SeaQuest)
    & \cite{Dove_2021}
    \\
    \midrule
    ATLAS $W,Z$ 7 TeV ($\mathcal{L}=4.6$~fb$^{-1}$)
    & \cite{Aaboud:2016btc}
    \\
    ATLAS DY 2D 8 TeV
    & \cite{Aaboud:2017ffb}
    \\
    ATLAS high-mass DY 2D 8 TeV
    & \cite{Aad:2016zzw}
    \\
    ATLAS $\sigma_{W,Z}$ 13 TeV
    & \cite{Aad:2016naf}
    \\
    ATLAS $W^+$+jet 8 TeV
    & \cite{Aaboud:2017soa}
    \\
    ATLAS $\sigma_{tt}^{\rm tot}$ 13 TeV ($\mathcal{L} = \SI{139}{\per\femto\barn}$)
    & \cite{Aad:2020tmz}
    \\
    ATLAS $t\bar{t}$ lepton+jets 8 TeV
    & \cite{Aad:2015mbv}
    \\
    ATLAS $t\bar{t}$ dilepton 8 TeV
    & \cite{Aaboud:2016iot}
    \\
    ATLAS single-inclusive jets 8 TeV, R=0.6
    & \cite{Aaboud:2017dvo}
    \\
    ATLAS dijets 7 TeV, R=0.6
    & \cite{Aad:2013tea}
    \\
    ATLAS direct photon production 13 TeV
    & \cite{Aaboud:2017cbm}
    \\
    ATLAS single top $R_{t}$ 7, 8, 13 TeV
    & \cite{Aad:2014fwa,Aaboud:2016ymp,Aaboud:2017pdi}
    \\
    \midrule
    CMS dijets 7 TeV
    & \cite{Chatrchyan:2012bja}
    \\
    CMS 3D dijets 8 TeV
    & \cite{Sirunyan:2017skj}
    \\
    CMS $\sigma_{tt}^{\rm tot}$ 5 TeV
    & \cite{Sirunyan:2017ule}
    \\
    CMS $t\bar{t}$ 2D dilepton 8 TeV
    & \cite{Sirunyan:2017azo}
    \\
    CMS $t\bar{t}$ lepton+jet 13 TeV
    & \cite{Sirunyan:2018wem}
    \\
    CMS $t\bar{t}$ dilepton 13 TeV
    & \cite{Sirunyan:2018ucr}
    \\
    CMS single top $\sigma_{t}+\sigma_{\bar{t}}$ 7 TeV
    & \cite{Chatrchyan:2012ep}
    \\
    CMS single top $R_{t}$ 8, 13 TeV
    & \cite{Khachatryan:2014iya,Sirunyan:2016cdg}
    \\
    \midrule
    LHCb $Z\to \mu\mu, ee$ 13 TeV
    & \cite{Aaij:2016mgv}
    \\
    \bottomrule
\end{tabular}
    \end{center}
    \caption{
        Observables included in the test data. We wish to stress that the observable
        central values themselves are not used, however the experimental
        uncertainities are used in the definition of the closure estimators, and
        the corresponding predictions from either the underlying law or the
        closure fits.
    }
    \label{tab:summarise_new_data}
\end{table}

For completeness, in Fig.~\ref{fig:DataKinematicCoverage}
the kinematic coverage of the training datasets, which
as mentioned is the NNPDF3.1-like dataset used in \cite{Faura_2020}, and the
test datasets shown in Tab.~\ref{tab:summarise_new_data} is plotted.

\begin{figure}
    \centering
    \includegraphics[width=0.8 \textwidth]{plot_xq2.png}
    % TODO: make PDF for higher res image.
    \caption{The kinematic coverage of the training and test data
    used to train the models and produce results presented in this paper. We emphasise
    that the split of datasets was largely chosen on practical grounds, not because
    of a deep reason to split the data chronologically. The kinematics of the two
    sets of data with this particular split overlaps but there are also kinematic
    regions which the test dataset probes, for which there was no training data.}
    \label{fig:DataKinematicCoverage}
\end{figure}

\section{Understanding $\deltachi$}

In the closure test presented in NNPDF3.0 \cite{nnpdf30} there was a data-space
estimator which aimed to measure the level of over or under fitting, $\deltachi$.
Here we discuss how $\deltachi$ can emerge from the bias-variance decomposition
and then use the linear model to try and understand it in the context of
viewing the ensemble of model replicas as a sample from the posterior distribution
of the model given the data.

Despite the link between the estimators emerging from the decomposition of
$\eout$ and the posterior distribution for data which is not used to inform
the model parameters, if we perform the same decomposition as in
Sec.~\ref{sec:ClosureEstimatorsDerivation} but set
$\testset{\obspriorcent}=\obspriorcent$ then we find that the cross term
in the final line of Eq.~\ref{eq:EoutDecomposition} does not go to zero when
the expectation across data is taken because there is a dependence on
$\obspriorcent$ in both the model predictions and the noisey data. As a result
we have to modify Eq.~\ref{eq:ExpectedBiasVariance} to be
\begin{equation}\label{eq:ExpectedBiasVarianceTraining}
    \mathbf{E}_{\obspriorcent}[\ein] =
    \mathbf{E}_{\obspriorcent}[\bias] + 
    \mathbf{E}_{\obspriorcent}[\var] +
    \mathbf{E}_{\obspriorcent}[{\rm noise}] +
    \mathbf{E}_{\obspriorcent}[\noisecross]\, ,
\end{equation}
where we refer now to the right hand side of
Eq.~\ref{eq:ExpectedBiasVarianceTraining} as $\ein$ because it's evaluated on the
data used to inform the model replicas.

Now if we examine the definition of $\deltachi$ introduced
in~\cite{Ball:2014uwa}, defined as the difference between the
$\chi^2$ between the expectation value of the model predictions and the level
one data, and the $\chi^2$ between the underlying observable values and the
level one data. In~\cite{Ball:2014uwa} the denominator was also set to be the
second term in the numerator, however here we slightly re-define
$\deltachi$ to instead simply be normalised by the number of data points:
\begin{equation}\label{eq:deltachi2def1}
    \begin{split}
        \deltachi &= \frac{1}{\ndata} \left[
            \left( \emodel{\fwdobsop\left(\modelvecrep\right)} - \obspriorcent \right)^T
            \obspriorcov^{-1}
            \left( \emodel{\fwdobsop\left(\modelvecrep\right)} - \obspriorcent \right) - 
            \left( \law - \obspriorcent \right)^T
            \obspriorcov^{-1}
            \left( \law - \obspriorcent \right)
        \right] \\
        &= \bias + \noisecross \, ,
    \end{split}
\end{equation}
where in the second line we show how $\deltachi$ itself can be decomposed to
be equal to two of the terms in Eq.~\ref{eq:ExpectedBiasVarianceTraining}.

Constant values of $\deltachi$ define elliptical contours in data space
centered on the level one data. $\deltachi = 0$, in particular, defines a
contour which is centered on the level one data and passes through the
underlying law. When viewing $\deltachi$ from a classifcal fitting perspective,
if $\deltachi < 0$ then the expectation value of the model
predictions fit the level one data better than the underlying observables -
which indicates an overfitting of the shift, $\boldsymbol{\shift}$. Similarly,
$\deltachi > 0$ indicates some underfitting of the level one data.

If we return to the linear model we can write the analytic value of
$\deltachi$. Firstly, since $\testset{\obspriorcent}=\obspriorcent$ we can
simplify Eq.~\ref{eq:BiasLinearModel}
\begin{equation}\label{eq:BiasLinearModelSimple}
    \begin{split}
        \mathbf{E}_{\obspriorcent}[{\rm bias}] &= \frac{1}{\ndata}
            {\rm Tr} \left[
                \linmap \modelpostcov \linmap^T \obspriorcov^{-1}
            \right] \\
            &= \frac{1}{\ndata}{\rm Tr} \left[ \modelpostcov \modelpostcov^{-1}\right] \\
            &= \frac{\nmodel}{\ndata} \, ,
    \end{split}
\end{equation}
because $\modelpostcov \modelpostcov^{-1}$ is an $\nmodel \times \nmodel$ identity
matrix. Similarly we can write down the cross term
\begin{equation}
    \begin{split}
        \mathbf{E}_{\obspriorcent}[\noisecross] &= \frac{-2}{\ndata} \mathbf{E}_{\obspriorcent} \left[
            (\linmap \modelpostcov \linmap^T \obspriorcov^{-1} \obsnoise)^T \obspriorcov^{-1} \obsnoise
        \right] \\
        &= \frac{-2}{\ndata}{\rm Tr} [\linmap \modelpostcov \linmap^T \obspriorcov^{-1}] \\
        &= -2 \frac{\nmodel}{\ndata}
    \end{split}
\end{equation}
which leaves us with
\begin{equation}
    \mathbf{E}_{\obspriorcent}[\deltachi] = - \frac{\nmodel}{\ndata} \, .
\end{equation}
The point is that the linear model has already been shown to be a sample from
posterior distribution of the model given the data. But from the classical
fitting point of view we would say this model has overfitted.

% From the
% Bayesian point of view, the usefulness of this estimator is less clear for
% several reasons:
% \begin{itemize}
%     \item The posterior model distribution cannot be treated as a prior for the
%     data it was obtained from, so the interpretation of estimators in
%     Eq.~\ref{eq:ExpectedBiasVarianceTraining} is unclear.
%     \item The posterior model distribution isn't fitted, it's just a result of
%     marginalising over the data.
%     \item The posterior model distribution for the linear model always appears
%     to be overfitting, which doesn't make so much sense given the previous
%     point and in light of Sec.~\ref{Sec:LinearMapEstimators} where the predictions
%     generated with the posterior model distribution are shown to have
%     faithful uncertainties.
% \end{itemize}
As such, we do not report any results with $\deltachi$ here, because
when $\biasvarratio = 1$, it doesn't add much to the discussion. It may still
be useful as a diagnostic tool when $\biasvarratio \neq 1$, which as discussed
could be for a variety of reasons - including fitting inefficiency. It also
may be used as a performance indicator for deciding between two fitting
methodologies: if both fits are shown to have $\biasvarratio = 1$, the
methodology with smaller magnitude of $\deltachi$ could be preferential.
The same could be said for bias and variance however, bias in particular is
clearly closely related to $\deltachi$.

\bibliographystyle{unsrt}
\bibliography{biblio}

\end{document}

%%% Local Variables:
%%% mode: latex
%%% TeX-master: t
%%% End:
