\section{NNPDF Monte Carlo approach to inverse problems}
\label{sec:closure-test}

In this section we will discuss the NNPDF approach to inverse problems, trying
to make contact explicitly with the formalism laid out in
Sec.~\ref{sec:inverse-problems}. In particular, Eq.~\eqref{eq:PosteriorModel}
gives a formal description of propagating our prior understanding of the data
into model space. In practice, sampling from the posterior distribution is
highly non-trivial.

\subsection{Fitting replicas}
\label{sec:fit-reps}

The approach for generating a sample in model space employed by NNPDF can
broadly be described as fitting model replicas to pseudo-data replicas. As
discussed in Eq.~\eqref{eq:NoisyInverseProblem} the experimental values are
subject to observational noise. If we assume this observational noise to be
multigaussian then the experimental central values, $\vv{\levone}$, are given
explicitly by
\begin{equation}
    \label{eq:levelonedata}
    \vv{\levone} = \vv{\law} + \vv{\shift},
\end{equation}
where $\vv{\law}$ is the 'true' underlying law and $\vv{\shift} \sim
\mathcal{N}(0, \cov_D)$ where $\cov_D$ is the experimental covariance matrix. In
Eq.~\eqref{eq:levelonedata}, each basis vector corresponds to a separate data
point, and the vector of shifts $\vv{\shift}$ permits correlations between data
points according to the covariance matrix provided by the experiments. Given the
data, the NNPDF approach is to compute the MAP estimator discussed in the
previous section, \ie\ finding the model that minimises the $\chi^2$ to the
data. The uncertainty on the MAP estimator is computed by bootstrapping the
data, \ie\ by generating an ensemble of pseudo-data, called replicas, that
fluctuate around $\vv{\levone}$. By fitting each replica, we obtain an ensemble
of models whose distribution is representative of the fluctuations of the MAP
estimator. The fitted pseudo-data is generated by augmenting the data with some
noise, $\noise^{\repind}$,
\begin{equation}
    \label{eq:leveltwodata1}
    \vv{\levtwo}^{\repind}
    = \vv{\levtwo}(\vv{\law},\,\vv{\shift},\,\vv{\noise}^{\repind})
    = \vv{\law} + \vv{\shift} + \vv{\noise}^{\repind},
\end{equation}
where the replica index $k$ refers to each replica having a noise vector drawn
independently from $\vv{\noise} \sim \mathcal{N}(0, \cov)$. The parameters for
each model replica maximise the likelihood of getting the corresponding
pseudo-data replicas from the model. This is a special case of MAP estimation,
described in Eq.~\eqref{eq:MAP}, where the model prior is uniform - in other
words there is no prior information in model space. In this framework, the
parameterisation of the model is fixed, so the model space, is the space of
parameters $u \in \mathbb{R}^{\nmodel}$. Furthermore, we actually find the
parameters which minimise the $\chi^2$ between the predictions from the model
and the corresponding pseudo-data $\levtwo^{\repind}$
\begin{equation}
    \begin{split}
        u^{\repind}_* &= \arg\min_{u^{\repind}} \repchis \\
        &= \arg\min_{u^{\repind}} \sum_{ij} \diffreptwo_i \invcov{ij} \diffreptwo_j\, ,
    \end{split}
    % \exp \left( - \frac{1}{2} \sum_{ij} \diffreptwo_i {(2\cov)}^{-1}_{ij} \diffreptwo_j \right)
\end{equation}
where we used the short-hand notation
\begin{equation}
    g^{\repind} = \mathcal{G}\left(u^{\repind}\right)
\end{equation}
for the theoretical prediction obtained from model $u^{\repind}$. As usual,
minimising the $\chi^2$ is equivalent to maximising the likelihood,
$\likelihood$, since $\chi^2 \equiv -\log{\likelihood}$.

\subsection{Fluctuations of fitted values}
\label{sec:fluct-fit-values}

It is not immediately obvious that our MC methodology, maximising the likelihood
on an ensemble of pseudo-data replicas, should guarantee that the model replicas
are indeed sampled from the posterior distribution of parameters given data as
described \eg\ in Eq.~\ref{eq:PosteriorModel}. In order to investigate this
issue, we will again consider a model, whose predictions are linear in the model
parameters, where the posterior distribution of model parameters can be written
explicitly. The forward map is given by
\begin{equation}\label{eq:LinForwardMap}
    \mathcal G(u) = X u\, ,
\end{equation}
where $X$ is some $\ndata\times\nmodel$ matrix. A practical example, which can
elucidate the following arguments would be a polynomial model. Then $u$ is a
vector of $\nmodel$ polynomial coefficients and $X$ is the Vandermonde matrix
\begin{equation}
    X =
    \begin{bmatrix}
        1  & x_1 & \ldots& x_1^{\nmodel-1} \\ 
        1  & x_2 & \ldots& x_2^{\nmodel-1} \\ 
        \vdots  & \vdots & \vdots& \vdots \\ 
        1  & x_{\ndata} & \ldots & x_{\ndata}^{\nmodel-1} 
    \end{bmatrix}.
\end{equation}
In this case the forward map yields
\begin{equation}
    \label{eq:PolyMod}
    y_I = \sum_{k=0}^{\nmodel-1} u_k x_I^k\, , 
\end{equation}
where $I=1,\ldots,\ndata$. The arguments here are not restricted to polynomials,
however, and apply to any model whose forward map can be expressed as
Eq.~\eqref{eq:LinForwardMap}, for example linear shallow approximation of neural
networks \cite{ADVANI2020428}. If the prior distribution of model parameters is
uniform then the posterior distribution of model parameters given the data is
\begin{equation}
    \label{eq:PosteriorMultiGaussModel}
    \begin{split}
        p(u | z) &\propto
        \exp \left( -\frac{1}{2} (Xu - z)^T \invcov{} (Xu - z)\right) \\
        &= \exp \left( -\frac{1}{2} (u - X^+z)^T X^T\invcov{}X (u - X^+z)\right)\, ,
    \end{split}
\end{equation}
where in the second line we are assuming that $X$ has linearly independent rows,
and therefore $X X^T$ is invertible. Under these hypotheses $X^+$ is a
right inverse and can be computed as
% can we restrict X to be real and just take transpose here?
\begin{equation}
    \label{eq:RightInverse}
    X^+ = X^T \left(X X^T\right)^{-1}\, .
\end{equation}
Eq.~\ref{eq:PosteriorMultiGaussModel} exposes that the posterior distribution of
the model parameters is multigaussian, with mean $\bar{u} = X^+z$ and covariance
$(X^T\invcov{}X)^+ = X^+ \cov (X^+)^T$. If instead we deploy the NNPDF Monte
Carlo method to fitting model replicas, then in the case under study
$\arg\min_{u^{\repind}} \repchis$ is found analytically when the derivative of
$\repchis$ with respect to the model parameters is zero, i.e.
\begin{equation}
    \begin{split}
        \label{eq:MAPEstLinModel}
        u^{\repind}_* &= (X^T\invcov{}X)^{+}
        \left( X^T \invcov{} \vv{\levone} + X^T \invcov{} \vv{\noise}^{\repind} \right), \\
        &= X^+ (\vv{\levone} + \vv{\noise}^{\repind}).
    \end{split}
\end{equation}
Eq.~\ref{eq:MAPEstLinModel} shows that $u_*$ is a linear combination of the
Gaussian variables $\epsilon$, and therefore is also a Gaussian variable. Its
probability density is then completely specified by the average and variance of
$u_*$, which can be calculated explicitly, given that the probability density
for $\epsilon$ is known.  In this way, we can show explicitly that under the
assumptions specified above, $u_* \sim \mathcal{N}( X^+z, X^+ \cov (X^T)^+)$.
In other words, when the model predictions are linear in the model parameters,
the NNPDF MC method is shown to produce a sample of models from the posterior
distribution of model parameters given the data.
%TODO: move here the narrow prior?

\subsection{Closure test}
\label{sec:closure-test}

The concept of the closure test, which was first introduced in
Ref.~\cite{nnpdf30}, is to construct artificial data by using a known
pre-existing function to generate the {\em true} observable values, $\vv{\law}$.
This is achieved by choosing $\lawmodel$ such that $\vv{\law} =
\mathcal{G}(\lawmodel)$. Then the experimental central values are artificially
generated according to Eq.~\ref{eq:levelonedata} and our assumption $\vv{\shift}
\sim \mathcal{N}(0, \cov)$. In \cite{nnpdf30}, $\vv{\law}$ is referred to as
level 0 (L0) data and $\vv{\levone}$ is referred to as level 1 (L1) data.
Finally, if we use the NNPDF MC method to fit artificially generated closure
data, we denote by $\levtwo$ the pseudo-data replicas that are fitted by the
model replicas. These pseudo-data replicas are referred to as level 2 (L2) data.

In a closure test, our assumption of the prior distribution of the data is
enforced. In the original closure test in NNDPF3.0 there was also no
modelisation uncertainty, the true observable values were assumed to be obtained
by applying the forward map $\mathcal G$ to a vector in model space $\lawmodel$.

It is worth noting that the assumption of zero modelisation uncertainties is
quite strong and likely unjustified in many areas of physics. In the context of
fitting parton distribution functions there are potentially missing higher order
uncertainties (MHOUs) from using fixed order perturbative calculations as part
of the forward map. MHOUs have been included in parton distribution fits
\cite{AbdulKhalek:2019ihb} and in the future these should be included in the
closure test, however this is beyond the scope of the study presented here,
since MHOUs are still not included in the NNPDF methodology by default. In the
results presented in the rest of this paper we do include nuclear and deuteron
uncertainties, as presented in \cite{Ball:2018twp, Ball:2020xqw}, since they are
to be included in NNPDF fits by default. Extensive details for including
theoretical uncertainties, modelled as theoretical covariance matrices can be
found in those references. For the purpose of this study the modelisation
uncertainty is absorbed into the prior of the data, since
\begin{equation}
    \levone = \mathcal{G}(u) + \shift + \delta
\end{equation}
where $\delta \sim \mathcal{N}(0, C^{\rm theory})$. But since the modelisation
uncertainty is independent of the data uncertainty then from the point of view
of a closure test we can absorb $\delta$ into $\shift$ by modifying the data
prior: $\shift \sim \mathcal{N}(0, C + C^{\rm theory})$, we must also update the
likelihood of the data given the model to use the total covariance $(C + C^{\rm
theory})$. From now onwards we will omit $C^{\rm theory}$ because it is
implicit that we always sample and fit data using the total covariance matrix
which includes any modelisation uncertainty we currently take into account as
part of our methodology.

% To summarise, the closure test presented here enforces the assumed prior of
% the data and that our assumed modelisation uncertainty properly accounts for
% all sources of modelisation uncertainty.

