\section{Polynomial model}
Here describe the definition of poly toy model and how to fit
\section{Neural Network Parton distribution functions}
% Should this section go later? I think so.
When fitting experimental data we vary the parameters of a set of PDF replicas
at the initial scale such that the $\chi^2$ is minimised between the
corresponding theory predictions and a generated pseudodata replica. A set of
PDFs usually refers to a set of seperate continuous functions, one for each
flavour of PDF in a particular basis. In this specific study, fits performed
with \nfit\ parameterise the set of PDFs as a single neural network which takes
as input $x$ and $\ln x$ and returns 8 outputs, one for each flavour in the
fitting basis, multiplied by some preproccessing exponents. The output for a
single flavour $j$ is
\begin{equation}
    NN(x, \ln x)_j * x^{1-\alpha_j} * (1-x)^{\beta_j},
\end{equation}
where each flavour has it's own preproccessing exponents $\alpha$ and $\beta$,
parameters that are varied in these fits, and $NN(x, \ln x)_j$ is the
$j^{\rm th}$ output from the neural network.
When an experiment is included in an NNPDF fit, we take the published
experimental central values and uncertainties (statistical and systematic)
and use these pieces of information to generate the pseudodata.
The pseudodata replica is generated
through Monte Carlo sampling by applying noise to the experimental
central values.
After fitting many sets of PDF replicas (usually of order 100 sets),
each set to an independently generated pseudodata replica, we have an ensemble of
PDF replicas.
The aim of this methodology, is to propagate the various sources of
uncertainty involved with fitting PDFs into the functional PDF space in a faithful
manner. This means that the distribution from which the PDF replicas are drawn
from should be representative of the probability distribution of the true
underlying PDF.

The closure test was introduced alongside NNPDF3.0 and will be described below.
At this stage it's important to note that the closure test may serve 3 purposes:
We can test whether or not the ensemble of PDF replicas does reflect the
probability distribution of the true underlying PDF; we can compare two different
fitting methodologies and use estimators to determine which one performs better;
we can try to understand different elements of our own methodology, such as
the different contributions to the PDF uncertainty. The PDF uncertainty will
be used as short-hand for referring to the distribution of replicas for a given
fit, we consider this the PDF uncertainty because if the distribution
of the true underlying PDF is reflected by the distribution of replicas for
a given fit, then the standard deviation of the replicas in PDF space or the
theory predictions obtained from those replicas in data space represents
the uncertainty of the prediction having performed a fit.