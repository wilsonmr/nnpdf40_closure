\section{Linear Forward Map}

\subsection{Closure estimators - Linear forward map}
\label{Sec:LinearMapEstimators}

Once again we return to the linear model framework set out in
Sec.~\ref{sec:fluct-fit-values}. We can perform an analytical closure
test in this framework, and check our
understanding of the closure estimators. Consider the true observable
values for the test data is given by
\begin{equation}\label{eq:LinearLawMap}
    \testset{\law} = \testset{\fwdobsop} \lawmodel
\end{equation}
where $\lawmodel \in \modelspace$, which means the number of (non-zero) parameters
in the underlying law is less than or equal to the number of parameters in the
model, $\nlaw \leq \nmodel$. Using the previous results from
Sec.~\ref{sec:fluct-fit-values}, we can write down the
difference between the true observables and the predictions from the MAP estimator
(or the expectation of the model predictions across model replicas - in the
linear model these are the same)
\begin{equation}
    \begin{split}
        \emodel{\testset{\fwdobsop}\left(\modelvecrep\right)} - \testset{\law} &=
        \testset{\linmap} (\modelpostcent - \lawmodel ) \\
        &= \testset{\linmap} \modelpostcov \linmap^T \obspriorcov \, \obsnoise \, ,
    \end{split}
\end{equation}
where we recall that $\linmap$ is the forward map to the training observables
and $\obspriorcent$ are
the corresponding training central values. Calculating the covariance across
training data of
$\emodel{\testset{\fwdobsop}\left(\modelvecrep\right)} - \testset{\law}$
gives
\begin{equation}
    \covcent = \testset{\linmap} \modelpostcov \testset{\linmap}^T \, ,
\end{equation}
so the full expression for $\mathbf{E}_{\obspriorcent}[{\rm bias}]$ is given by
\begin{equation}\label{eq:BiasLinearModel}
    \mathbf{E}_{\obspriorcent}[{\rm bias}] = \frac{1}{\ndata}
    {\rm Tr} \left[
        \testset{\linmap} \modelpostcov \testset{\linmap}^T
        \testset{\obspriorcov}^{-1}
    \right].
\end{equation}
We note that if the test data is identical the data the model was fitted on,
we recover an intuitive result $\mathbf{E}_{\obspriorcent}[{\rm bias}] = \frac{\nmodel}{\ndata}$.
Consider the example of the polynomial, the maximum value which $\nmodel$ can
take whilst $\linmap$ still has linearly independent rows is $\ndata$ and in this case
the $\mathbf{E}_{\obspriorcent}[{\rm bias}]$ takes it's maximum value of 1. The central
predictions from the model exactly pass through each data point.

We can perform a similar exercise on the model replica predictions. The difference
between the predictions from model replica $\repind$ and the expectation value
of the model predictions is
\begin{equation}
    \begin{split}
        \testset{\fwdobsop}\left(\modelvecrep\right) -
        \emodel{\testset{\fwdobsop}\left(\modelvecrep\right)} &=
        \testset{\linmap} (\modelvecrep - \modelpostcent) \\
        &= \testset{\linmap} \modelpostcov \linmap^T \obspriorcov \, \noise \, .
    \end{split}
\end{equation}
Since $\noise$ and $\obsnoise$ follow the same distribution, it's clear that
\begin{equation}
    \covrep = \covcent,
\end{equation}
which, as a result means that
\begin{equation}
    \var = \mathbf{E}_{\obspriorcent}[{\rm bias}].
\end{equation}
We recall that when the map is linear, the NNPDF MC methodology generates replicas
which are sampled from the posterior distribution of the model given the data.
We have shown here that provided the underlying law belongs to the model
space, the posterior distribution of the model predictions satisfy the
requirement that $\biasvarratio = 1$.

We note that due to the invariance of the trace under cyclic permutations, we
can rearrange Eq.~\ref{eq:BiasLinearModel} as
\begin{equation}
    \mathbf{E}_{\obspriorcent}[{\rm bias}] = \frac{1}{\ndata}
    {\rm Tr} \left[
        \modelpostcov
        \testset{\linmap}^T \testset{\obspriorcov}^{-1} \testset{\linmap}
    \right] \, ,
\end{equation}
where the term $\testset{\linmap}^T \testset{\obspriorcov}^{-1} \testset{\linmap}$
can be understood as the covariance matrix of the posterior distribution in model
space given the test data, with zero prior knowledge of the model \viz
\begin{equation}\label{eq:BiasTraceModelPost}
    \mathbf{E}_{\obspriorcent}[{\rm bias}] = \frac{1}{\ndata}
    % testset fails here due to double postscript
    {\rm Tr} \left[ \modelpostcov {\modelpostcov}^{ \prime -1} \right] \, ,
\end{equation}
where we emphasise that the covariance matrices $\modelpostcov$ are
$\testset{\modelpostcov}$ from completely independent Bayesian inferences
with no prior information on the model parameters, unlike in
Eq.~\ref{eq:ModelPostSequential} where a sequential marginalisation causes
$\testset{\modelpostcov}$ to depend on $\modelpostcov$.

Alternatively, if we perform a sequential marginalisation of
the data, and use the result in Eq.~\ref{eq:ModelPostSequential}, but
then take $\testset{\obspriorcov}^{-1} \to 0$, \ie\ there is no information
on the observables in the test set, then
\begin{equation}
    \modelpostcov^{-1} = \linmap^T \obspriorcov^{-1} \linmap \, ,
\end{equation}
or the total posterior model distribution, is identical to the posterior model
distribution given just the training data - as you would expect.
Now we can express bias (or variance) as
\begin{equation}\label{eq:BiasTraceObsPost}
    \mathbf{E}_{\obspriorcent}[{\rm bias}] = \frac{1}{\ndata}
    {\rm Tr} \left[
        \testset{\obspostcov}
        \testset{\obspriorcov}^{-1}
    \right] \, ,
\end{equation}
where $\testset{\obspostcov}$ is the covariance of the posterior distribution of
$\testset{\obs}$ with no prior information on that data. This might seem peculiar
because in determining $\testset{\obspostcov}$ we took the limit
$\testset{\obspriorcov}^{-1} \to 0$, because we had no prior information on the
unseen data, however
in Eq.~\ref{eq:BiasTraceObsPost} we require
$\testset{\obspriorcov}^{-1}$ to be finite.
However
we rationalise Eq.~\ref{eq:BiasTraceObsPost} as a comparison between
the posterior distribution in the space of data of some unseen observables to an
independently determined prior from performing the relevant experiment which
measures the same observables. Comparing moments of these two distributions
is what you would expect when the new experimental data is published.

% If we marginalise
% instead over $\testset{\obs}$, then
% \begin{equation}
%     \mathbf{E}_{\obspriorcent}[{\rm bias}] = \frac{1}{\ndata}
%     {\rm Tr} \left[
%         \obspostcov
%         \obspriorcov^{-1}
%     \right] \, ,
% \end{equation}
% but as we already know, if there was no prior information on the model
% this reduces to $\frac{\nmodel}{\ndata}$ (to convince yourself, set
% $\testset{\modelpostcov} = \modelpostcov$ in Eq.~\ref{eq:BiasTraceModelPost}).

\subsection{Underparameterised model}

Note that if we were to choose the
number of model parameters such that $\nlaw > \nmodel$, then the variance
would be unaffected, since the underlying law parameters cancel. However, the
bias would now contain an extra term from the extra parameters in the
underlying law, schematically:
\begin{equation}
    \begin{split}
        (\emodel{\testset{\fwdobsop}\left(\modelvecrep\right)} - \testset{\law})_i =
        \sum_{1 \leq j \leq \nmodel} \testset{\linmap}_{ij} (\modelpostcent - \lawmodel)_j -
        \sum_{\nmodel < j \leq \nlaw} \testset{\linmap}_{ij} \lawmodel_j,
    \end{split}
\end{equation}
which would mean that $\biasvarratio \neq 1$. This demonstrates that requiring
$\biasvarratio = 1$ demands that the model space is suitably flexible, if the
underlying law is parameterised then this can be summarised as requiring
$\lawmodel \in \modelspace$. Note that in the
underparameterised regime the model replicas are still drawn from the posterior
distribution, however because $\lawmodel \notin \modelspace$ we've somehow
invalidated the assumptions that go into the relation between model predictions
and the data-space prior.

Although $\biasvarratio$ was largely chosen on practical
grounds, we see that it is still a stringent test that our assumptions are
correct and that the distribution our model replicas are drawn from is meaningful,
this is what we mean when we say {\em faithful uncertainties}.

An unfortunate
trade-off when using $\biasvarratio$ is it can't be used as a diagnostic
tool, and is instead used simply for validation. For example, if
$\biasvarratio > 1$, then we
can't know whether there was a problem with the fitting methodology used to
generate the model replicas or a deeper issue such as an inflexible model.
